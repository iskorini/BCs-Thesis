%--------------------------------------------------------------
% thesis.tex 
%--------------------------------------------------------------
% Corso di Laurea in Informatica 
% http://if.dsi.unifi.it/
% @Facolt\`a di Scienze Matematiche, Fisiche e Naturali
% @Universit\`a degli Studi di Firenze
%--------------------------------------------------------------
% - template for the main file of Informatica@Unifi Thesis 
% - based on Classic Thesis Style Copyright (C) 2008 
%   Andr\'e Miede http://www.miede.de   
%--------------------------------------------------------------
\documentclass[twoside,openright,titlepage,fleqn,
headinclude,12pt,a4paper,BCOR5mm,footinclude]{scrbook}
%--------------------------------------------------------------
\newcommand{\myItalianTitle}{Estensione del linguaggio FACPL per esprimere politiche di accesso alle risorse di un sistema di calcolo basate sul comportamento passato\xspace}
\newcommand{\myEnglishTitle}{Extension of FACPL language for expression of access policy to resource of a system
based on the past behaviour\xspace}
% use the right myDegree option
\newcommand{\myDegree}{Corso di Laurea in Informatica\xspace}
%\newcommand{\myDegree}{
	%Corso di Laurea Specialistica in Scienze e Tecnologie 
	%dell'Informazione\xspace}
	\newcommand{\myName}{Federico Schipani\xspace}
	\newcommand{\myProf}{Rosario Pugliese\xspace}
	\newcommand{\myOtherProf}{Andrea Margheri\xspace}
	\newcommand{\myFaculty}{
	Scuola di Scienze Matematiche, Fisiche e Naturali\xspace}
	\newcommand{\myUni}{\protect{
	Universit\`a degli Studi di Firenze}\xspace}
	\newcommand{\myLocation}{Firenze\xspace}
	\newcommand{\myTime}{Anno Accademico 2014-2015\xspace}
	\newcommand{\myVersion}{Version 0.1\xspace}
%--------------------------------------------------------------
\usepackage[italian]{babel}
\usepackage[utf8x]{inputenc} 
\usepackage[T1]{fontenc} 
\usepackage[square,numbers]{natbib} 
\usepackage[fleqn]{amsmath}  
\usepackage{ellipsis}
\usepackage{listings}
\usepackage{subfig}
\usepackage{caption}
\usepackage{appendix}
\usepackage{siunitx}
\usepackage{float}

%--------------------------------------------------------------
\usepackage{dia-classicthesis-ldpkg}
%--------------------------------------------------------------
% Options for classicthesis.sty:
% tocaligned eulerchapternumbers drafting linedheaders 
% listsseparated subfig nochapters beramono eulermath parts 
% minionpro pdfspacing
\usepackage[eulerchapternumbers,linedheaders,subfig,beramono,eulermath,
parts]{classicthesis}
%--------------------------------------------------------------
\newlength{\abcd} % for ab..z string length calculation
% how all the floats will be aligned
\newcommand{\myfloatalign}{\centering} 
\setlength{\extrarowheight}{3pt} % increase table row height
\captionsetup{format=hang,font=small}



%--------------------------------------------------------------
% Comandi personali
%--------------------------------------------------------------
\graphicspath{{Chapters/Image/}}
\newcommand{\MyFigure}[3]{
	\begin{figure}[H]

		\centering
		\includegraphics[width=#3\linewidth]{#1}
		\caption{ #2 }\label{fig:#1}
		
	\end{figure}
}

\newcommand{\MyFig}[4]{
	\begin{figure}[#4]

		\centering
		\includegraphics[width=#3\linewidth]{#1}
		\caption{ #2 }\label{#1}
		
	\end{figure}
}

\newcommand{\statusattribute}{\textit{Status Attribute}}
\newcommand{\status}{\textit{Status}}
\input{syntax_original/macro}
\renewcommand{\lstlistingname}{Codice}
\renewcommand{\lstlistlistingname}{Lista di Codici}


% Define Language
\lstdefinelanguage{FACPL}
{
  % list of keywords
  morekeywords={
    policy,
    rule,
    target,
    equal, {less-than}, obl, pep, pdp, status
  },
  sensitive=false, % keywords are not case-sensitive
  morecomment=[l]{//}, % l is for line comment
  morecomment=[s]{/*}{*/}, % s is for start and end delimiter
  morestring=[b]" % defines that strings are enclosed in double quotes
}

%--------------------------------------------------------------
% Layout setting
%--------------------------------------------------------------
\usepackage{geometry}
\geometry{
a4paper,
ignoremp,
bindingoffset = 1cm, 
textwidth     = 13.5cm,
textheight    = 21.5cm,
	lmargin       = 3.5cm, % left margin
	tmargin       = 4cm    % top margin 
	}

	\lstset{
	frame=tb,
	language=Java,
	aboveskip=3mm,
	belowskip=3mm,
	showstringspaces=false,
	columns=flexible,
	basicstyle={\small\ttfamily},
	numbers=none,
	breaklines=true,
	breakatwhitespace=true,
	tabsize=3
	}
%--------------------------------------------------------------
\begin{document}
\frenchspacing
\raggedbottom
\pagenumbering{roman}
\pagestyle{plain}
%--------------------------------------------------------------
% Frontmatter
%--------------------------------------------------------------
\include{titlePage}
\pagestyle{scrheadings}
%--------------------------------------------------------------
% Mainmatter
%--------------------------------------------------------------
\pagenumbering{arabic}
% use \cleardoublepage here to avoid problems with pdfbookmark
%\include{intro} % use \myChapter command instead of \chapter
\tableofcontents
\listoffigures
\cleardoublepage
\thispagestyle{empty}
\begin{flushright}
\null\vspace{\stretch {1}}
%\emph{Ezechiele 25:17. “Il cammino dell'uomo timorato è minacciato da ogni parte dalle iniquità degli esseri egoisti e dalla tirannia degli uomini malvagi. Benedetto sia colui che nel nome della carità e della buona volontà conduce i deboli attraverso la valle delle tenebre, perché egli è in verità il pastore di suo fratello e il ricercatore dei figli smarriti. E la mia giustizia calerà sopra di loro con grandissima vendetta e furiosissimo sdegno su coloro che si proveranno ad ammorbare e infine a distruggere i miei fratelli. E tu saprai che il mio nome è quello del Signore quando farò calare la mia vendetta sopra di te.” \break --- Jules Winnfield [Voce di Luca Ward]} \vspace{\stretch{2}}\null
\emph{"Stay hungry, stay hungry" \break --- Paolo Bitta, l'uomo chiamato contratto} \vspace{\stretch{2}}\null
\end{flushright}
\cleardoublepage
\myChapter{Introduzione}
FACPL è un linguaggio basato su XACML, supportato da una libreria scritta in Java. Rispetto a XACML la sintassi di FACPL è molto più semplice e concisa quindi permette di formalizzare in modo facile, non ambiguo e rapido politiche di \textit{access control}. FACPL però non godeva di caratteristiche per consentire di fare richieste ed ottenere risposte valutando anche il comportamento passato, quindi l'obiettivo di questa tesi è stato estenderlo in modo da implementare nuove e fondamentali funzionalità che permettono di sfruttare questo nuovo modello chiamato \textit{Usage Control}. \\ \par
La questione sul controllo agli accessi è un problema molto rilevante a cui si è provato a trovare una soluzione in questi ultimi anni. Negli ultimi decenni, a partire dal 1970, sono stati proposti molti modelli come ACL, RBAC, ABAC e PBAC, ognuno ha i suoi punti di forza ed i suoi punti deboli.
\textit{Access Control List} è un modello basato su liste di accesso associate ad ogni risorsa che definiscono le regole di accesso, è un modello molto semplice da implementare, ma quando i dati erano eccessivi diventava poco efficiente. Successivamente è stato introdotto un modello basato su ruoli, il quale associava ad insiemi di risorse dei ruoli, ovvero dei gruppi di utenti raggruppati secondo una caratteristica comune. Un punto debole di questo modello è la mancata specificità riguardante la definizione delle regole che rende difficile assegnare permessi particolari a singole risorse. A seguire è stato introdotto un modello basato su attributi dove le decisioni vengono prese valutando caretteristiche del richiedente, della risorsa e dell'ambiente in cui è stata fatta la richiesta. Un difetto di quest'ultimo modello è che non offre una buona scalabilità. Per porre rimedio a questo difetto è stato introdotto un modello basato su politiche, che oggi risulta essere uno dei più utilizzati, e che si basa su politiche, ovvero un insieme di regole su attributi.
Recentemente, Sandhu e Park, hanno introdotto un'estensione di \textit{Access Control} chiamata \textit{Usage Control} le cui caratteristiche sono la possibilità di prendere decisioni durante l'accesso e di basarsi sul comportamento passato in fase di valutazione di una richiesta. Il che lo rende molto adatto ad ambienti come il Web, Cloud o comunque legati in qualche modo alla rete. \\ \par
Il resto della tesi è organizzato in questo modo. Nel capitolo~\ref{cap:accessControl} vengono analizzati in maggior dettaglio tutti i modelli di \textit{Access Control} descritti precedentemente ed inoltre viene dedicata una sezione a \textit{Usage Control} dove vengono anche introdotti due esempi che verranno portati avanti durante gli altri capitoli. Nel capitolo~\ref{cap:facpl} è descritto FACPL allo stato precedente alla stesura di questo documento, ovvero quando ancora non era possibile esprimere politiche di accesso basate sul comportamento passato. Alla fine del capitolo verrà anche proposto un esempio e sarà spiegato perché non è possibile prendere decisioni di \textit{Usage Control}. Nel capitolo~\ref{cap:usagecontrolfacpl} viene trattata l'estensione di FACPL ad un livello sintattico e semantico, spiegando quali nuove componenti sono state introdotte e analizzandone il loro significato ed utilizzo in un contesto reale. Usando le nuove a fine capitolo sono stati riproposti in FACPL gli esempi citati in sezione~\ref{sec:usage_control}. Nell'ultimo capitolo, il ~\ref{cap:estensione_libreria}, invece sono trattati gli argomenti visti in quello precedente ma dal punto di vista dell'estensione della libreria Java.

\myChapter{Access Control}
Ai giorni d'oggi esistono moltissimi sistemi capaci di condividere dati e risorse computazionali, 
ed impedire accessi non autorizzati è diventata una priorità inderogabile.
Per esempio molti dati personali possono essere raccolti durante alcune attività quotidiane, e proteggere questi dati
da malintenzionati è molto importante.
Questo, e moltre altre ragioni, sono il motivo per cui esistono sistemi di Access Control, ovvero dei sistemi definiti da un insieme di condizioni 
che permettono di creare una prima linea difensiva contro accessi indesiderati.
\section{Storia dell'Access Control}
Negli anni sono stati proposti diversi approcci per cercare di definire un modello efficiente e scalabile.

\myChapter{Formal Access Control Policy Language}
Negli anni molti linguaggi sono stati proposti per definire policy di access control. Uno di questi è stato rilasciato nel 2003 da parte di OASIS ed il suo nome è \textit{eXtensible Access Control Markup Language} (XACML). Questo linguaggio ha una sintassi basata su XML e fornisce caratteristiche avanzate per l'access control. Il problema fondamentale di XACML è che non ha una sintassi facile da leggere e da scrivere. \\
L'obiettivo di \textit{Formal Access Control Policy Language} (FACPL) è definire una sintassi alternativa per XACML in modo da renderlo più agevole da usare.
FACPL quindi è parzialmente inspirato a XACML, ma oltre ad introdurre una nuova sintassi ridefinisce alcuni aspetti aggiungendo nuove caratteristiche. Il suo scopo però non è sostituire XACML, ma fornire un linguaggio compatto ed espressivo per facilitare le tecniche di analisi attraverso tool specifici.

\section{Il processo di valutazione di FACPL}
\label{sec:valutazione_facpl}


\MyFigure{FACPL_EVALUATION.jpg}{Il processo di valutazione di FACPL}{1}
In figura \ref{fig:FACPL_EVALUATION.jpg} è mostrato il processo di valutazione delle policy definite in FACPL.
I componenti principali sono tre:
\begin{itemize}
\item{Policy Repository (PR)}
\item{Policy Decision Point (PDP)}
\item{Policy Enforcement Point (PEP)}
\end{itemize}
Le policy sono memorizzate nel PR, il quale le rende disponibili al PDP che deciderà se garantire l'accesso o meno (Primo step).
Nello step 2, quando il PEP riceve una richiesta, le credenziali di quest'ultima vengono codificate in una sequenza di attributi (ogni attributo è una coppia stringa valore) che, nello step 3, andranno a loro volta a formare una \textit{FACPL Request}.
Al quarto step il \textit{context handler} aggiungerà attributi di ambiente (per esempio l'ora di ricezione della richiesta) e manderà la richiesta al PDP.
A questo punto il PDP, tra il quinto e l'ottavo step, valuterà la richiesta e fornirà un risultato, il quale può eventualmente contenere delle \textit{obligations}.
La decisione del PDP può essere di quattro tipi, \textit{permit}, \textit{deny}, \textit{not-applicable} o \textit{indeterminate}.
Il significato delle prime due decisioni è facilmente intuibile, mentre per le ultime due vuol dire che c'è stato un errore durante la valutazione.
Gli errori possono essere di diverso tipo, e vengono gestiti attraverso algoritmi che combinano le decisioni delle varie policy per ottenere un risultato finale.
Le \textit{obligations} sono azioni, eseguite dal PEP, correlate al sistema di controllo degli accessi. Queste azioni possono essere di svariati tipi, come per esempio generare un file di log, o mandare una mail.
Allo step 13, sulla base del risultato delle \textit{obligations}, il PEP esegue un processo chiamato\textit{Enforcement} il quale restituirà un'altra decisione.
Quest'ultima decisione corrisponde alla decisione finale del sistema e può differire da quella del PDP.


\section{La sintassi di FACPL}
\label{sec:facpl_syntax}

\input{syntax_original/facpl_syntax}

La sintassi di FACPL è definita nella tabella \ref{tab:facpl_syntax}.
La sintassi è fornita come una grammatica di tipo EBNF, dove il simbolo ? corrisponde ad un elemento opzionale, il simbolo $*$ corrisponde ad una sequenza con un numero arbitrario di elementi (anche 0), ed il simbolo $+$ corrisponde ad una sequenza non vuota con un numero arbitrario di elementi.\\
Al livello più alto c'è il \textit{Policy Authorisation System (PAS)}, il quale definisce le specifiche del PEP e del PDP.
Il PEP è definito semplicemente come un \textit{enforcing algorithm} che sarà applicato per decidere quali decisioni verrà eseguito il processo di \textit{enforcement}. \\
Il PDP invece è definito come una sequenza (non vuota) di \textit{Policy}, ed un algoritmo di combining che combinerà i risultati di queste policy per ottenere un unico risultato finale.\\
Una \textit{policy} può essere una semplice \textit{rule} o una \textit{policy set}, quest'ultima avrà al suo interno altre \textit{policy set} o \textit{rule}, ed in questo modo viene formata una gerarchia di policy.\\
Un \textit{policy set} individua un target, che è una espressione che indica il set di richieste di accesso alla quale si applica la policy, una lista di \textit{obligations}, che definiscono azioni obbligatorie o opzionali che devono essere eseguite nel processo di \textit{enforcement}, una sequenza di altre \textit{policy}, ed un algoritmo per combinarle.\\
Una \textit{rule} includerà un \textit{effect}, che sarà permit o deny quando la regola è valutata correttamente, un target ed una lista di \textit{obligations}.\\
Le \textit{Expressions} sono formate da \textit{attribute names} e valori (per esempio boolean, double, strings, date).\\
Un \textit{Attribute Name} indica il valore di un attributo il quale può essere contenuto nella richiesta o nel contesto. FACPL usa per gli \textit{Attribute Name} una forma del tipo \textit{Identifier / Identifier }, dove il primo Identifier indica la categoria, ed il secondo il nome dell'attributo.
Per esempio \textit{Action / ID} rappresenta il valore di un attributo ID di categoria Action.\\
I \textit{Combining Algorithm} implementano diverse strategie che servono per risolvere conflitti tra le varie decisioni, restituendo alla fine un'unica decisione finale.\\
Una \textit{obligation} ha al suo interno un effect, un tipo, ed una azione eseguita dal PEP con la relativa \textit{Expression}.\\
Una \textit{request} consiste di una sequenza di attributi organizzati in categorie.\\
La risposta ad una valutazione di una richiesta FACPL è scritta usando la sintassi riportata in tabella \ref{tab:facpl_context_syntax}.
La valutazione in due step, descritta precedentemente in sezione~\ref{sec:valutazione_facpl}, produce due tipi di risultati. Il primo è la risposta del PDP, il secondo è una decisione, ovvero una risposta del PEP.
La decisione del PDP, nel caso in cui ritorni \texttt{permit} o \texttt{deny}, viene associata ad una lista, anche vuota, di fulfilled obligations.\\
Una \textit{fulfilled obligation} è una semplice coppia formata da un tipo (M o O) ed una azione i quali argomenti sono ottenuti dalla valutazione del PDP.

\section{La semantica di FACPL}

Molteplici sono le componenti di FACPL, e la semantica ora verrà informalmente analizzata.\cite{fullfacpl}
Prima verrà presentato il processo che porterà ad una risposta del PDP, successivamente il processo di enforcement del PEP.\\
Quando il PDP riceve una richiesta, per prima cosa valuta la richiesta sulle basi delle policy disponibili, successivamente determinerà un risultato combinando le decisioni ritornate da queste policy attraverso degli algoritmi di combining.\\
La valutazione della policy rispetto alla richiesta comincia verificando l'applicabilità alla richiesta, che è fatta valutando un espressione definita \textit{target}.\\
Si possono valutare due casi distinti:
\begin{itemize}
\item[-] Supponiamo che l'applicabilità dia esito positivo, nel caso ci sia una \textit{rule} sarà ritornato il valore risultato dalla valutazione di quest'ultima, mentre se c'è un \textit{policy set} il risultato è ottenuto valutando le policy contenute all'interno, e combinando i loro valori con un algoritmo specificato in fase di creazione del PDP. Successivamente a queste valutazioni verrà effettuato il fulfilment delle obligation contenute all'interno delle policy.
\item[-] Supponiamo ora che l'applicabilità non dia esito positivo, ovvero la valutaizone del \textit{target} restituisca \texttt{false}. In questo caso il risultato della policy sarà \texttt{not-app}. Mentre se \textit{target} restituisce un valore non booleano o ritorna un errore il risultato della policy sarà \texttt{indet}.
\end{itemize}
Valutare le espressioni corrisponde ad applicare degli operatori e risolvere i nomi degli attributi che contengono, e di conseguenza ricavarne un valore.\\
Se non è possibile trovare un attributo, magari perché non esiste, viene ritornato un valore speciale, chiamato \texttt{BOTTOM}. Questo valore può essere usato per implementare diverse strategie per gestire l'assenza di attributi. FACPL gestisce questo valore come una specie di \texttt{false}, quindi permette la mancanza di attributi senza la generazione di errori.\\
La valutazione di un espressione tiene conto anche dei tipi degli argomenti. Se l'argomento è del tipo aspettato l'operatore viene applicato correttamente, sennò, se un argomento è \texttt{BOTTOM} e nessun'altro è \texttt{error} viene ritornato \texttt{BOTTOM}, mentre se almeno uno di essi è \texttt{error}, viene ritornato \texttt{error}.\\
Con l'operatore \texttt{and} o \texttt{or} il trattamento sarà leggermente dievrso, in quanto \texttt{BOTTOM} viene ritornato solo se un argomento è tale e nessun'altro è \texttt{false} o \texttt{error}, mentre in caso contrario viene ritornato \texttt{error}.\\
La valutazione di una policy termina con il fulfillment di tutte le \texttt{obligations} le quali hanno il valore di applicabilità coincidente con quello ritornato dalla valutazione della policy. Quest'operazione consisten nel valutare tutte le espressioni presenti al interno delle \texttt{obligations} coinvolte nel processo. Se ci sarà un errore nel processo di fulfilment allora il risultato della policy sarà \texttt{indet}, altrimenti il risultato del fulfilment sarà uguale a quello della valutazione del PDP.\\
Gli algoritmi di combining, come detto prima hanno lo scopo di combinare le decisioni risultanti dalla valutazione delle richieste in accordo con le policy. Un'altra funzione che hanno è ritornare le \textit{obligations} corrette nel caso in cui la valutazione finale risulti \texttt{permit} o \texttt{deny}. Questa famiglia di algoritmi ha una strategia $\delta$ che viene usata per restituire le \textit{obligation}, e può essere di due tipi.
Il primo tipo è la strategia \texttt{all} (tutto), ovvero richiede la valtuazione di tutte le policy e ritorna le \texttt{fulfilled obligation} pertinenti a tutte le decisioni.\\
Il secondo tipo è la strategia \texttt{greedy} (golosa) prescrive che appena è ottenuta una decisione che non può cambiare a causa della valutazione di susseguenti policy nella sequenza di input, l'esecuzione si arresta.\\
Come ultimo step il risultato del PDP viene mandato al PEP per l'enforcement.
Il PEP per effettuare questo processo deve eseguire l'azione all'interno di ogni \texttt{fulfilled obligation} e decidere come comportarsi per le decisioni di tipo \texttt{not-app} e \texttt{indet.}\\
Per fare questo processo usa delle strategie. In particolare, l'algoritmo \texttt{deny-biased} (rispettivamente, \texttt{permit-based}) effettua l'enforcement dei \texttt{permit} (rispettivamente \texttt{deny}) solo quando tutte le corrispondenti obligations sono correttamente scaricate, mentre effettua l'enforcement dei \texttt{deny} (rispettivamente \texttt{permit}) in tutti gli altri casi. Invece, l'algoritmo di base lascia tutte le decisioni non cambiate ma, in caso di decisioni \texttt{permit} e \texttt{deny}, effettua l'enforcement di \texttt{indet} se un errore occorre quando si stanno rilasciando le \texttt{obligations}. Questo evidenzia che le \texttt{obligations} non solo influenzano il processo di autorizzazione, ma anche l'enforcement. Gli errori causati dalle \texttt{obligations} con tipo O vengono ignorati.

\section{Esempi con FACPL}
\label{sec:esempi_facpl}

\myChapter{Implementare Usage Control in FACPL}
\label{cap:usagecontrolfacpl}

\ac{FACPL}, per come è descritto nel Capitolo~\ref{cap:facpl}, non supporta Usage Control, di conseguenza non è possibile prendere decisioni basate sul comportamento passato. Grazie a delle nuove strutture, implementate insieme al mio collega Filippo Mameli, è possibile prendere questo tipo di decisioni.

Questa estensione ha richiesto delle modifiche alla sintassi del linguaggio in modo da poter sfruttare facilmente le nuove funzionalità. Introdurre queste modifiche ha richiesto del lavoro sulla libreria, in quanto è stato necessario aggiungere nuove componenti e di conseguenza modificare il processo di valutazione delle policy.

In Sezione~\ref{sec:estensione_del_processo_di_valutazione} viene analizzato il nuovo processo di valutazione alla luce delle modifiche introdotte in \ac{FACPL}.
Nella Sezione~\ref{sec:estensione_linguistica} invece viene discussa l'estensione dal punto di vista della sintassi, introducendo la nuova grammatica.
Successivamente, in Sezione~\ref{sec:semantica} viene spiegata la semantica dei nuovi costrutti implementati. Infine in Sezione~\ref{sec:esempi} sono proposti in \ac{FACPL} due case study già presentati in Sezione \ref{sec:casi_studio}.

\section{Estensione del processo di Valutazione} % (fold)
\label{sec:estensione_del_processo_di_valutazione}
Il processo di valutazione, presentato in Sezione~\ref{sec:valutazione_facpl}, è stato esteso per via delle modifiche introdotte. 
Rispetto al processo di valutazione standard, sono state aggiunte
componenti al grafico, rendendolo così adatto allo \textit{Usage Control}, e quindi assicurare un controllo continuativo basato sul comportamento passato.
\begin{figure}[h]
 \centering 
	\includegraphics[scale = 0.49]{./Chapters/Image/evalvect.pdf}
 \caption{Valutazione dopo lo stato}
 \label{fig:evalStatus}
\end{figure}

Come si nota in Figura \ref{fig:evalStatus} è stato aggiunto un componente alla struttura della valutazione.
Questo componente rappresenta lo \status \ (Stato), al quale il \ac{PDP} e \ac{PEP} ci accedono tramite attributi.
Questi attributi vengono chiamati \statusattribute. \par
Analizziamo quindi, a scopo esemplificativo, come viene gestita la presenza dello stato. Inizialmente viene definito il sistema, che ora rispetto a quelli già citati in Sezione~\ref{sec:valutazione_facpl}, ha un componente in più, ovvero lo Stato.
Fino al quarto step il comportamento è analogo a quello precedente, mentre cambia negli step successivi. \par
Al quinto step il \textit{PDP} non necessiterà solo dei normali attributi d'ambiente, ma necessiterà anche degli \statusattribute \ coinvolti nella richiesta effettuata. Il \textit{Context Handler} quindi non andrà solo a fare la ricerca all'interno dell'environment, ma andrà a cercare anche gli \statusattribute \ all'interno dello \status. \par
A questo punto, quando il \ac{PDP} avrà tutte le informazioni necessarie si potrà passare alla vera e propria valutazione della richiesta che avviene come sempre. \par
Nel caso in cui viene restituto \permit \ o \deny \ è necessario fare l'enforcement della risposta del \ac{PDP}. Questo processo differisce dal precedente poiché ora sono state implementate nuove azioni sullo stato che devono essere eseguite dal \ac{PEP} (Passo 11-14). Le nuove azioni sono eseguite attraverso un nuovo tipo di obligations, chiamate \textit{ObligationStatus}. 
Quest'ultime vengono valutate dal \ac{PEP} alla pari di una normale Obligation, ma la sostanziale differenza tra esse ed una normale Obligation è legata alla funzione che contengono. Mentre le normali Obligation conterranno generiche funzioni, come creare un log o mandare una mail, le Obligation Status potranno eseguire azioni per modificare lo stato del sistema.
Una volta effettuato l'enforcement viene restituta la decisione finale.



% section estensione_del_processo_di_valutazione (end)

\section{Estensione Linguistica} % (fold)
\label{sec:estensione_linguistica}
Per implementare queste nuove funzionalità è stata modificata anche la grammatica di \ac{FACPL}.
Nella grammatica estesa sono state aggiunte nuove regole di produzione e simboli terminali che 
codificano le nuove funzionalità. \par
\input{syntax_original/facpl_new_syntax}
Come è facilmente osservabile dalla consultazione della Tabella~\ref{tab:facpl_new_syntax} le aggiunte rispetto alla tabella riporta in Sezione~\ref{sec:facpl_syntax} sono state diverse, vediamo adesso quali sono. \par
La prima modifica è nel \ac{PAS}, cioè  lo \status, che è della forma $$status: Attribute$$ ed è formato da uno o più \textit{Attribute}. \par
Passiamo ora a descrivere \textit{Attribute} che è della forma $$(Type\ Identifier (= Value)^?)$$
questo tipo particolare di attribute, che è lo \statusattribute \ descritto in precedenza, è formato innanzitutto da un \textit{Type}, dopo il tipo è richiesta una generica stringa chiamata \textit{Identifier}, che sarà un generico nome da dare all'attributo, infine viene richiesto un \textit{Value}, ovvero un valore, che in questo caso è opzionale, all'atto pratico vuol dire che l'attributo di stato potrà essere inizializzato con un valore oppure potrà essere solamente definito, lasciando che il valore sia quello di default.

\textit{Type} è il tipo che avrà l'attributo di stato, e potrà essere \texttt{int, boolean, date o double}. \par
La regola \textit{PepAction} è stata modificata in modo tale che includesse nuove funzioni per operare sugli attributi di stato.
Infine l'ultima regola di produzione modificata è stata quella riguardante \textit{Attribute Names}, in questo caso è stata semplicemente aggiunto, a fianco di \textit{Identifier/Identifier}, una nuova produzione Status/\textit{Identifier}. Questa nuova produzione serve semplicemente per permettere il confronto tra attributi di stato attraverso le già esistenti \textit{Expression}.
La sintassi delle risposte è rimasta invariata.

% section estensione_linguistica (end)

\section{Semantica} % (fold)
\label{sec:semantica}
La semantica di FACPL rimane molto simile a quella descritta in Sezione \ref{sec:semantica_originale}, quindi verranno di seguito descritte in modo informale solo le novità introdotte. \par
La prima di queste riguarda la valutazione delle richieste dal \ac{PDP}. Il \ac{PDP} ora non si deve più basare solo su richieste totalmente scollegate l'una dall'altra, e quindi è stato introdotto il concetto di \status.
Lo stato permette di rappresentare il comportamento passato del sistema, e lo fa introducendo una nuova serie di attributi chiamati \statusattribute.\par
Nel linguaggio questo nuovo tipo di attributi viene considerato al pari di normali attributi, quindi si ha la possibilità di effettuare tutte le operazioni di confronto tra di essi, ma in più si deve avere la possibilità di modificarli e memorizzarli in modo da poterli sfruttare per \textit{Usage Control}. \par

Per questo sono state aggiunte delle \textit{Pep Action}, ovvero delle azioni eseguite dal \ac{PEP} in seguito alla valutazione di \textit{Obligations status}. La prima di queste è l'addizione, 
 e permette, in seguito alla valutazione di una \textit{Obligations}, l'aggiunta di un valore numerico, definito dallo sviluppatore, ad uno \statusattribute \ di tipo \texttt{double} o \texttt{int}. 
\begin{verbatim}
 obl:
     [permit M add(counter, 2)]
\end{verbatim}
Per esempio l'esecuzione di questa \textit{Obligation} su un'attributo, chiamato \textit{counter}, inizializzato a $0$, porterà l'attributo al 
valore 2. 

L'operazione di somma è stata implementata anche per altri due tipi, \texttt{Date} e \texttt{String}.
Oltre all'addizione sono presenti funzioni per la sottrazione, divisione e moltiplicazione che operano in modo analogo a questa appena descritta, ma sono definite soltanto su tipi numerici.

Un'altra operazione implementata modifica il valore originale di uno \statusattribute \ con il valore passatogli come secondo parametro.
\begin{verbatim}
 obl:
     [permit M flag(isFoo, true)]
\end{verbatim}
L'esecuzione con successo di questa \textit{Obligation} porterà l'attributo \textit{flag} ad avere un valore \texttt{true}. Questo tipo di operazione è stata definita anche per il tipo \texttt{Date} e \texttt{String}.

Vediamo ora un esempio di questa nuova estensione, prenderemo spunto dal primo caso trattato in precedenza nella Sezione~\ref{sec:estensione_del_processo_di_valutazione}.
%SOSTITUIRE CON LISTINGS PER FACPL
\lstinputlisting[language = FACPL, caption = {Esempio per la sintassi}\label{lst:esempio_sintassi}]{./Source/first_example_facpl}
In questo esempio (Codice \ref{lst:esempio_sintassi}) si può vedere come nel PAS è stato definito uno stato, con al suo interno uno solo attributo inizializzato con valore $0$.
Successivamente si può notare nella \textit{Rule} che viene fatto un controllo sul valore di quest'attributo.
Infine nella \textit{Obligation} si può notare come viene aggiornato lo stato dell'attributo in base al risultato della valutazione della \textit{Rule}.

\section{Formalizzazione dei case study} % (fold)
\label{sec:esempi}

Queste nuove funzionalità introdotte servono allo scopo descritto in Sezione~\ref{sec:usage_control}, ovvero l'implementazione di un nuovo modello chiamato \textit{Usage Control}.
In questi due semplici case study lo stato gioca un ruolo fondamentale, in quanto il sistema di access control, per poter soddisfare requisiti di consistenza deve tener traccia del comportamento passato.
Mostreremo ora l'implementazione in \ac{FACPL} dei due esempi trattati in Sezione~\ref{sec:casi_studio}. Per comodità è usata la sintassi del plugin di Eclipse, che differisce leggermente da quella presentata in \ref{sec:estensione_linguistica}. 

\subsection{Accesso ai file} % (fold)
\label{ssub:primo_esempio}
Il primo esempio in Sezione~\ref{sec:casi_studio} poneva una regola sull'accesso ai file, ovvero permetteva un massimo di due persone in contemporanea che potevano effettuare l'accesso in lettura oppure un massimo di una persona che poteva ottenere l'accesso in scrittura. \par
Tutte le policy che saranno mostrate in questa sezione sono incluse in un \textit{PolicySet} che racchiude al suo interno un espressione di tipo target, mostrata in Codice~\ref{lst:PrimoEsempio_FACPL_target}.
\lstinputlisting[language = FACPL, linerange = 1-2,  caption = {Target PolicySet}\label{lst:PrimoEsempio_FACPL_target}]{./Source/EsempioReadWrite_facpl.fpl}
Il target del PolicySet \texttt{ReadWrite\_Policy} verifica che le richieste provengano da utenti che hanno nome \textit{Alice} o \textit{Bob}, in caso contrario il responso sarà \texttt{Not Applicable}. \par
Successivamente sono state scritte quattro policy per gestire le quattro operazioni possibili, ovvero \texttt{read}, \texttt{write}, \texttt{stopRead} ed infine \texttt{stopWrite}. \par
Prendiamo in considerazione la policy mostrata in Codice~\ref{lst:PrimoEsempio_FACPL_write}, ovvero quella per la \texttt{write}. Come prima è presente un target, che richiede questa volta due diverse condizioni, la prima riguarda l'id del file richiesto, la seconda invece richiede che l'azione sia \texttt{write}. Le parti interessanti di questa policy sono due, la prima riguarda la \textit{Rule}, la seconda la \textit{Obligation}. \par
La \textit{Rule} restituisce \texttt{permit} se le due condizioni dell'equal sono vere, come si può facilmente notare l'operazione di confronto non viene fatta tra una stringa ed un normale attributo, ma tra una stringa ed uno \statusattribute. \par
L'ultima cosa da notare è l'unica \textit{Obligation} presente per questa policy. Questo tipo particolare di \textit{Obligation}, ha sempre al suo interno un'azione che verrà eseguita dal PEP, questa volta però non sarà una semplice azione come scrivere un log o mandare una mail, l'azione andrà a modificare lo stato del sistema, mettendo il valore \texttt{true} all'attributo \textit{isWriting}.
\lstinputlisting[language = FACPL, linerange = 4-13, firstnumber = 4, caption = {Policy Write}\label{lst:PrimoEsempio_FACPL_write}]{./Source/EsempioReadWrite_facpl.fpl}
Successivamente si prende in considerazione la policy per l'operazione \texttt{stopWrite}, mostrata in Codice~\ref{lst:PrimoEsempio_FACPL_stopwrite}.
\lstinputlisting[language = FACPL, linerange = 24-32, firstnumber = 24, caption = {Policy StopWrite}\label{lst:PrimoEsempio_FACPL_stopwrite}]{./Source/EsempioReadWrite_facpl.fpl}
Il target definito da questa policy è molto simile a quello precedente, la differenza è nella seconda parte: in questo caso l'azione richiesta non è \texttt{Write}, ma \texttt{StopWrite}.\par
Come prima c'è una \textit{Rule} all'interno che esegue un confronto tra uno \statusattribute\ ed un valore, in questo caso \texttt{true}. Questo confronto in questo caso serve per verificare la reale presenza di uno scrittore al momento della richiesta. 
Nel caso fosse presente, e quindi la regola restituisse \texttt{true}, viene eseguita la Obligation Status che si occupa della modifica dello stato.
Le altre policy, per definire le restanti due operazioni, sono analoghe a quelle appena descritte e si possono trovare in Codice~\ref{lst:PrimoEsempio_FACPL}.
\subsubsection{Valutazione}
Prendiamo ora una serie di richieste ed analizziamone la loro valutazione.
\lstinputlisting[language = FACPL, caption = {Richieste del primo esempio}\label{lst:PrimoEsempioRichieste_FACPL}]{./Source/EsempioReadWrite_facpl_richieste.fpl}
La prima richiesta proviene da Alice, e sarà una lettura sul file1, la successiva proviene da Bob, è sempre sul file1, ma l'azione richiesta è di scrittura. Le altre sono analoghe.\par

L'output di queste richieste è mostrato in Tabella~\ref{tab:risultati_1}. Analizziamo ora il motivo di queste decisioni. Nella prima richiesta ovviamente nessuno sta leggendo o scrivendo, quindi viene tranquillamente restituito \permit. Visto che è presente una \textit{obligation} lo stato verrà aggiornato, sommando un'unità al contatore di letture.
\begin{table}[H]
\centering
\footnotesize
\caption{Risultati della valutazione}
\label{tab:risultati_1}
\begin{tabular}{cccc}
\multicolumn{1}{l}{} & \textbf{Risultato} & \textbf{Stato Prima} & \textbf{Stato dopo} \\ \hline
\textbf{Richiesta 1} & \textit{PERMIT} & \begin{tabular}[c]{@{}c@{}}isWriting = false\\ CounterReadFile1 = 0\end{tabular} & \begin{tabular}[c]{@{}c@{}}isWriting = false\\ CounterReadFile1 = 1\end{tabular} \\ \hline
\textbf{Richiesta 2} & \textit{DENY} & \begin{tabular}[c]{@{}c@{}}isWriting = false\\ CounterReadFile1 = 1\end{tabular} & \begin{tabular}[c]{@{}c@{}}isWriting = false\\ CounterReadFile1 = 1\end{tabular} \\ \hline
\textbf{Richiesta 3} & \textit{PERMIT} & \begin{tabular}[c]{@{}c@{}}isWriting = false\\ CounterReadFile1 = 1\end{tabular} & \begin{tabular}[c]{@{}c@{}}isWriting = false\\ CounterReadFile1 = 2\end{tabular} \\ \hline
\textbf{Richiesta 4} & \textit{PERMIT} & \begin{tabular}[c]{@{}c@{}}isWriting = false\\ CounterReadFile1 = 2\end{tabular} & \begin{tabular}[c]{@{}c@{}}isWriting = false\\ CounterReadFile1 = 1\end{tabular} \\ \hline
\textbf{Richiesta 5} & \textit{PERMIT} & \begin{tabular}[c]{@{}c@{}}isWriting = false\\ CounterReadFile1 = 1\end{tabular} & \begin{tabular}[c]{@{}c@{}}isWriting = false\\ CounterReadFile1 = 0\end{tabular} \\ \hline
\textbf{Richiesta 6} & \textit{PERMIT} & \begin{tabular}[c]{@{}c@{}}isWriting = false\\ CounterReadFile1 = 0\end{tabular} & \begin{tabular}[c]{@{}c@{}}isWriting = true\\ CounterReadFile1 = 0\end{tabular} \\ \hline
\end{tabular}
\end{table}
Alla seconda richiesta l'utente richiede la scrittura, che gli viene negata perché c'è già qualcuno che sta leggendo, però lo stesso utente effettua un'altra richiesta, questa volta in lettura, che gli viene concessa.
La quarta e la quinta richiesta vengono fatte per avvisare il sistema che la lettura è terminata, ovviamente la risposta è \permit, e la \textit{obligation} corrispondente decrementerà il contatore.
La sesta ed ultima richiesta è una scrittura, che questa volta viene permessa, poiché nessuno sta scrivendo o leggendo.




\subsection{Noleggio e acquisto di contenuti} % (fold)
\label{sub:secondo_esempio}

In questo secondo esempio analizzeremo il caso di un'azienda di distribuzione di contenuti multimediali che vuole regolare l'accesso di quest'ultimi attraverso policy.
Faremo un breve esempio con un solo file e due utenti, uno dei due utenti comprerà il file, l'altro lo noleggierà a tempo determinato.
Nel codice~\ref{lst:SecondoEsempio_FACPL} vengono mostrate solo una parte delle policy presenti nel Codice completo ~\ref{lst:SecEsCompl} \ mostrato in Appendice~\ref{cap:appendiceA}.
\lstinputlisting[firstline = 1, lastline = 39, language = FACPL, caption = {Secondo Esempio}\label{lst:SecondoEsempio_FACPL}]{./Source/second_example_facpl}
Queste due policy, e anche le altre che non sono state mostrate, sono racchiuse tutte all'interno del \textit{Policy Set} Negozio il quale come prima cosa verifica se chi ha fatto la richiesta ha un determinato nome, in questo caso \textit{Bob} o \textit{Alice}. \par
Successivamente, se uno dei due effettua la richiesta di \texttt{BUY}, ovvero l'acquisto senza alcun tipo di limitazione, si entra nella prima policy e, tramite le \textit{Obligation} si cambia l'attributo di stato. Invece se un utente decidesse di effettuare il noleggio con la modalità dove si limita il numero di visioni si entrerebbe nella seconda \textit{Policy Set} la quale, attraverso \textit{Obligations} aumenterà il numero di visioni di due unità. Analogo è il caso del noleggio a tempo. \par
Per disciplinare la visione è presente un altro \textit{Policy Set}, mostrato anch'esso parzialmente in codice~\ref{lst:secondo_esempio_view}.
\lstinputlisting[firstline = 59, lastline = 71, language = FACPL, caption = {Secondo Esempio}\label{lst:secondo_esempio_view}]{./Source/second_example_facpl}
\subsubsection{Valutazione}
Mostriamo ora in Codice~\ref{lst:richieste_2} alcune richieste che possono essere fatte al sistema ed analizziamo le risposte che produrranno, in Tabella~\ref{tab:valutazione_2} è mostrato un quadro riassuntivo del risultato delle richieste.
\lstinputlisting[language = FACPL, caption = {Richieste del Secondo Esempio}\label{lst:richieste_2}]{./Source/second_example_request}
La prima e la seconda richiesta sono richieste di visione, che ovviamente restituiranno entrambe \deny, in quanto lo stato del sistema non è stato modificato da nessuno poiché nè Alice nè Bob hanno effettuato acquisti o noleggi.
Successivamente Alice effettuerà un acquisto, e quindi tramite la Obligation Status verrà modificato lo stato del sistema, accreditando così l'acquisto. Dopo aver effettuato questa richiesta Alice ne effettua un'altra, questa volta di visione. 
A questo punto la policy che disciplina quest'ultima richiesta di Alice effettua una verifica dello \statusattribute \ \textit{AccessTypeAlice}, e visto che lo trova cambiato dalla precedente richiesta di acquisto permette la visione.
Bob, a cui all'inizio era stata negata la visione effettuerà una richiesta di noleggio e quindi cambierà lo stato. Dopo, sempre Bob, richiede la visione, il risultato di entrambe sarà ovviamente \permit.
% subsection secondo_esempio (end)

\begin{table}[H]
\centering
\footnotesize
\caption{Riassunto Valutazione}
\label{tab:valutazione_2}
\begin{tabular}{cccc}
\multicolumn{1}{l}{} & \textbf{Risultato} & \textbf{Stato Prima} & \textbf{Stato dopo} \\ \hline
\textbf{Richiesta 1} & \textit{DENY} & \begin{tabular}[c]{@{}c@{}}AccessTypeAlice = null\\ AccessTypeBob = null\end{tabular} & \begin{tabular}[c]{@{}c@{}}AccessTypeAlice = null\\ AccessTypeBob = null\end{tabular} \\ \hline
\textbf{Richiesta 2} & \textit{DENY} & \begin{tabular}[c]{@{}c@{}}AccessTypeAlice = null\\ AccessTypeBob = null\end{tabular} & \begin{tabular}[c]{@{}c@{}}AccessTypeAlice = null\\ AccessTypeBob = null\end{tabular} \\ \hline
\textbf{Richiesta 3} & \textit{PERMIT} & \begin{tabular}[c]{@{}c@{}}AccessTypeAlice = null\\ AccessTypeBob = null\end{tabular} & \begin{tabular}[c]{@{}c@{}}AccessTypeAlice = BUY\\ AccessTypeBob = null\end{tabular} \\ \hline
\textbf{Richiesta 4} & \textit{PERMIT} & \begin{tabular}[c]{@{}c@{}}AccessTypeAlice = BUY\\ AccessTypeBob = null\end{tabular} & \begin{tabular}[c]{@{}c@{}}AccessTypeAlice = BUY\\ AccessTypeBob = null\end{tabular} \\ \hline
\textbf{Richiesta 5} & \textit{PERMIT} & \begin{tabular}[c]{@{}c@{}}AccessTypeAlice = BUY\\ AccessTypeBob = null\end{tabular} & \begin{tabular}[c]{@{}c@{}}AccessTypeAlice = BUY\\ AccessTypeBob = TIME\\ BobFile1Expiration = 2016/04/22\end{tabular} \\ \hline
\textbf{Richiesta 6} & \textit{PERMIT} & \begin{tabular}[c]{@{}c@{}}AccessTypeAlice = BUY\\ AccessTypeBob = TIME\\ BobFile1Expiration = 2016/04/22\end{tabular} & \begin{tabular}[c]{@{}c@{}}AccessTypeAlice = BUY\\ AccessTypeBob = TIME\\ BobFile1Expiration = 2016/04/22\end{tabular} \\ \hline
\end{tabular}
\end{table}


% section esempi (end)

\myChapter{Conclusioni}
\label{cap:conclusioni}
Durante questa tesi è stata affrontato il lavoro di implementazione di \textit{Usage Control} in \ac{FACPL}.
Come primo compito ci siamo occupati di analizzare i principali modelli dedicati all'\textit{Access Control} e successivamente è seguita una fase di approfondimento sul modello \textit{Usage Control} proposto da Sandhu e Park.  \par
Il lavoro è seguito con una disamina sul linguaggio \ac{FACPL} in modo da comprendere al meglio la sintassi, la sematica e soprattutto il processo di valutazione così da avere un background e sapere come, e dove, intervenire per implementare del concetto di \status \ e di tutte le cose che conseguentemente ne derivano da esso, come nuovi attributi, nuove obligation e funzioni per operare su attributi. \par
Nella sintassi estesa sono state aggiunte nuove regole di produzione e ne sono state modificate alcune. Quelle modificate includono la definizione del sistema, mentre quelle aggiunte riguardano nuove funzioni, ed un nuovo tipo di attributo, chiamato \statusattribute.
Successivamente è stato svolto del lavoro sulla libreria Java per implementare queste nuove caratteristiche. \par
Nel capitolo \ref{cap:estensione_libreria} viene descritta l'implementazione delle nuove caratteristiche del linguaggio in Java. All'inizio del Capitolo viene descritta l'implementazione dello \status \ e di conseguenza degli \statusattribute. Successivamente sono è stata necessaria  l'implementazione  delle funzioni per la modifica di questi attributi in modo da poter cambiare lo stato con l'avanzare delle valutazioni delle richieste. L'estensione ha coinvolto anche il \ac{PEP}in quanto deve valutare un nuovo tipo di Obligation, ovvero le Obligation Status.
La differenza con le Obligation normali risiede nel fatto che le Obligation Status possono eseguire le funzioni per la modifica dello stato mostrate all'inizio del capitolo.
Durante la valutazione di una richiesta devono essere valutati anche gli attributi di stato. Permettere la valutazione di quest'ultimi da parte delle funzioni già esistenti è stato abbastanza semplice, poiché è bastato estendere la classe che implementa il contesto in modo tale che facesse la ricerca degli attributi anche all'interno dello stato.

\section{Sviluppi futuri}
\label{sec:futuro}

Gli esempi mostrati durante questa tesi sono basilari, e fondamentalmente sono stati scritti con il solo scopo di provare il funzionamento delle nuove funzionalità di \ac{FACPL}. Potrebbe risultare interessante applicare \ac{FACPL} a casi di studio reali, in modo da poterne verificare le potenzialità sul campo.\par
Durante lo sviluppo sono stati implementati solo alcuni tipi di dato e relative funzioni su di essi, in futuro sarebbe facilmente possibile implementarne di nuovi in quanto la libreria è stata
progettata per favorirne la rapida e semplice estendibilità. Per esempio potrebbe essere stimolante l'implementazione di un tipo nuovo come le Liste e funzioni come ricerca all'interno di esse, aggiunta ed eliminazione di elementi o il conteggio del numero di elementi.
\begin{thebibliography}{99}

\bibitem{NISTACM}{NIST - \emph{A survey of access Control Models} - \url{http://csrc.nist.gov/news_events/privilege-management-workshop/PvM-Model-Survey-Aug26-2009.pdf}}


\bibitem{UsageControlCloud}{Aliaksandr Lazouski, Gaetano Mancini, Fabio Martinelli, Paolo Mori - \emph{Usage Control in Cloud Systems} - Istituto di informatica e Telematica, Consiglio Nazionale delle Ricerche. }


\bibitem{FacplGuide}{FACPL Site - \url{http://facpl.sourceforge.net/guide/facpl_guide.html}}

\bibitem{fullfacpl}{Andrea Margheri, Massimiliano Masi, Rosario Pugliese, Francesco Tiezzi - \emph{A Formal Framework for Specification, Analysis and Enforcement of Access Control Policies}}

\bibitem{Park1}{Jaehong Park, Ravi Sandhu - \emph{Towards usage control models: beyond traditional access control}}

\bibitem{Park2}{Jaehong Park - \emph{Usage control: A unified framework for next generation access control} - Tesi di dottorato, George Mason University }

\bibitem{nistsite}{ Eric Chabrow - \emph{NIST Guide Aims to Ease Access Control} - http://www.bankinfosecurity.com/nist-publication-aims-to-ease-access-control-a-6612/op-1}



\bibitem{JavaCite}{Java Platform, Standard Edition 8 API Specification
 - \url{https://docs.oracle.com/javase/8/docs/api/}}

\bibitem{Grammatica}{ ISO/IEC 14977 - \emph{Information technology - Syntactic metalanguage - Extended BNF} }

\bibitem{Xtext}{Xtext Documentation - \url{https://eclipse.org/Xtext/documentation/}}

\end{thebibliography}

%--------------------------------------------------------------
\end{document}
%--------------------------------------------------------------
