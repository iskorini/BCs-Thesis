

\newcommand{\sr}[1]{#1}

\newcommand{\define}{\triangleq}



\newcommand{\modif}[1]{{\color{red}#1}}

%---------------------------------------
%ACRONYM
%---------------------------------------
\newcommand{\xacml}{\ac{XACML}}
\newcommand{\facpl}{\ac{FACPL}}
\newcommand{\pbac}{\ac{PBAC}}
\newcommand{\rbac}{\ac{RBAC}}

\newcommand{\pdp}{\ac{PDP}}
\newcommand{\pep}{\ac{PEP}}
\newcommand{\pap}{\ac{PAP}}
\newcommand{\pip}{\ac{PIP}}
\newcommand{\pr}{\ac{PR}}

\newcommand{\dsl}{\ac{DSL}}
\newcommand{\ide}{\ac{IDE}}

%%%%%%%%%%%%%%%%%%%%%%%%%%%%%
%SYNTAX
%%%%%%%%%%%%%%%%%%%%%%%%%%%%%
%-----------------------------------
%Algorithm
%-----------------------------------

\newcommand{\denyOver}{\x{d}\textrm{-}\x{over}}
\newcommand{\permitOver}{\x{p}\textrm{-}\x{over}}
\newcommand{\permitUnless}{\x{p}\textrm{-}\x{unless}\textrm{-}\x{d}}
\newcommand{\denyUnless}{\x{d}\textrm{-}\x{unless}\textrm{-}\x{p}}
\newcommand{\onlyOneApp}{\x{one}\textrm{-}\x{app}}
\newcommand{\firstApp}{\x{first}\textrm{-}\x{app}}
\newcommand{\weakCon}{\x{weak}\textrm{-}\x{con}}
\newcommand{\strongCon}{\x{strong}\textrm{-}\x{con}}


\newcommand{\denyOverO}[1]{\x{d}\textrm{-}\x{over}_{\x{#1}}}
\newcommand{\permitOverO}[1]{\x{p}\textrm{-}\x{over}_{\x{#1}}}
\newcommand{\permitUnlessO}[1]{\x{p}\textrm{-}\x{unless}\textrm{-}\x{d}_{\x{#1}}}
\newcommand{\denyUnlessO}[1]{\x{d}\textrm{-}\x{unless}\textrm{-}\x{p}_{\x{#1}}}
\newcommand{\onlyOneAppO}[1]{\x{one}\textrm{-}\x{app}_{\x{#1}}}
\newcommand{\firstAppO}[1]{\x{first}\textrm{-}\x{app}_{\x{#1}}}
\newcommand{\weakConO}[1]{\x{weak}\textrm{-}\x{con}_{\x{#1}}}
\newcommand{\strongConO}[1]{\x{strong}\textrm{-}\x{con}_{\x{#1}}}

%Obligation Types
\newcommand{\obM}{\modif{\x{M}}}
\newcommand{\obO}{\modif{\x{O}}}


%PEP
\newcommand{\based}{\x{base}}
\newcommand{\debugAlg}{\x{debug}}
\newcommand{\denyBiased}{\x{deny}\textrm{-}\x{biased}}
\newcommand{\permitBiased}{\x{permit}\textrm{-}\x{biased}}

%-------------------------------------
%Syntax formatting, comparison functions, examples
%-------------------------------------

\newcommand{\alice}{\x{Alice}}
\newcommand{\bob}{\x{Bob}}

\newcommand{\polSet}[4]{{\bf{\{}}#1{\,}\x{target:}\,#2 {?}\, \x{policies:}#3^{+} {\,}\x{obl:}#4^{*}\, {\bf{\}}}}
\newcommand{\polSetE}[4]{{\bf{\{}}#1 \, \x{target:}\,#2 \, \x{policies:}#3 \, \x{obl:}#4 \, {\bf{\}}}}
\newcommand{\pol}[4]{{\bf{\langle}}#1{\,}\x{target:}\,#2 {?}{\,}\, \x{rules:}#3^{+} {\,}\x{obl:}#4^{*}{\,}{\bf{\rangle}}}
\newcommand{\polE}[4]{{\bf{\langle}}#1{\,}\x{target:}\,#2 {\,}\, \x{rules:}#3 {\,}\x{obl:}#4 {\,}{\bf{\rangle}}}

\newcommand{\pdpPol}[2]{\{ #1 \,  \, #2 \}}
\newcommand{\obl}[1]{\, PepAction( #1^* ) }

\newcommand{\streq}{\x{equal}}
\newcommand{\exprOp}{\x{eop}}
\newcommand{\exprOperator}{\x{op}}

\newcommand{\attribute}[2]{( #1 , #2 )}

%\newcommand{\mEl}{m}
%\newcommand{\mEval}[2]{#2 \models #1}

\newcommand{\x}[1]{{\sf #1}}


%---------------------------------------
% Separator
%---------------------------------------

\newcommand{\set}[1]{\{#1\}} % brackets for sets
\newcommand{\Sep}{\ \mid\ }

\newcommand{\policySetBegin}{{\bf{\{}}}
\newcommand{\policyBegin}{{\bf{\{}}}
\newcommand{\policySep}{\,\bf{\ }\,}
\newcommand{\policiesBegin}{\x{policies:}\,}
\newcommand{\targetBegin}{\x{target:}\,}
\newcommand{\targetEnd}{\,{\bf{\ }}}
\newcommand{\policiesEnd}{\,{\bf{\ }}}
%\newcommand{\rulesBegin}{\x{rules:}\,}
%\newcommand{\rulesEnd}{\,{\bf{\ }}}
\newcommand{\policyEnd}{{\bf{\}}}}
\newcommand{\policySetEnd}{{\bf{\}}}}
%\newcommand{\conditionBegin}{\x{condition:}\,}
%\newcommand{\conditionEnd}{\,{\bf{\ }}}
\newcommand{\oblsBegin}{\x{obl:}\,}
\newcommand{\oblsEnd}{\bf{\ }\,}
\newcommand{\oblBegin}{{\bf{[}}}
\newcommand{\oblEnd}{{\bf{]}}}

\newcommand{\match}[3]{#1{\bf(}#2{\bf,}#3{\bf)}}
\newcommand{\ruleOpt}[1]{{\bf{(}}#1{\bf{)}}}
\newcommand{\oblOpt}[1]{{\bf{(}}#1{\bf{)}}}

\newcommand*\lfrac[2]{{}_{#1}\!\backslash\!^{#2}} % table combining algorithm

%------------------------------
%Decisions & values
%------------------------------
\newcommand{\permit}{\x{permit}}
\newcommand{\deny}{\x{deny}}
\newcommand{\notApp}{\x{not}\textrm{-}\x{app}}
\newcommand{\indet}{\x{indet}}
\newcommand{\excpt}{\perp}
\newcommand{\err}{\x{error}}

\newcommand{\indetD}{\x{indetD}}
\newcommand{\indetP}{\x{indetP}}
\newcommand{\indetDP}{\x{indetDP}}

\newcommand{\true}{\x{true}} 
\newcommand{\false}{\x{false}} 

%%%%%%%%%%%%%%%%%%%%%%%
%SEMANTICS
%%%%%%%%%%%%%%%%%%%%%%%

%Auxiliary Semantic Functions and notations
\newcommand{\pepSemR}[1]{(\!( #1 )\!)}
\newcommand{\concat}{{\bullet}}
\newcommand{\subobeff}[2]{#1\!\!\mid_{#2}} %sub sequence of obligations

%Name of Semantic Function
\newcommand{\denSemF}[1]{\mathcal{ #1 }}

%Semantic Brakets
%\newcommand{\denSem}[2]{[\![ #1 ]\!]_{#2}}
\newcommand{\denSem}[2]{[\![ #1 ]\!] #2}

%Named Semantics Functions
\newcommand{\policySem}[2]{\denSemF{P}\denSem{#1}{#2}}
\newcommand{\algSem}[2]{\denSemF{A}\denSem{#1}{#2}}
\newcommand{\exprSem}[2]{\denSemF{E}\denSem{#1}{#2}}
\newcommand{\oblSem}[2]{\denSemF{O}\denSem{#1}{#2}}
\newcommand{\oblSemS}[2]{\denSemF{OS}\denSem{#1}{#2}}
\newcommand{\pdpSem}[2]{\denSemF{P}dp\denSem{#1}{#2}}
\newcommand{\pepSem}[2]{\denSemF{E}A\denSem{#1}{#2}}
\newcommand{\pasSem}[1]{\denSemF{P}as\denSem{#1}{}}
\newcommand{\reqSemS}[2]{\denSemF{R}\denSem{#1}{#2}}  %two arguments, \name as second argument
\newcommand{\reqSem}[1]{\denSemF{R}\denSem{#1}{}} %used for PAS semantics

%Generic Element of each Syntactic Category
\newcommand{\expr}{\mathit{expr}}
\newcommand{\effect}{\mathit{e}}
\newcommand{\ob}{\mathit{o}}
\newcommand{\obType}{\mathit{t}}
\newcommand{\fo}{\mathit{f\!o}}
\newcommand{\foS}{\mathit{f\!o}^*}
\newcommand{\rSyntax}{\mathit{req}} %element of syntactic category Request
\newcommand{\req}{\mathit{r}} % element of functional representation of Request
\newcommand{\policy}{\mathit{p}}
\newcommand{\algSyntax}{\mathit{a}}
\newcommand{\pdpRes}{\mathit{res}}
\newcommand{\dec}{\mathit{dec}} %decision
\newcommand{\double}{\mathit{d}} %double
\newcommand{\extVal}{\mathit{w}} % Obligations and table of expression semantics
\newcommand{\enfAlg}{\mathit{ea}}
\newcommand{\pdpSyntax}{\mathit{pdp}}
\newcommand{\name}{\mathit{n}}
\newcommand{\val}{\mathit{v}}
\newcommand{\pepAction}{\mathit{pepAct}}

\newcommand{\algNT}{\mathit{Alg}} %used in syntax section
\newcommand{\algName}{\mathsf{alg}} % used in combining algorithms semantic section

%Semantics of Combiing Algorithms
\newcommand{\alg}[1]{\mathsf{alg}_{#1}}
\newcommand{\algD}{\alg{\delta}}
\newcommand{\algOp}{\otimes \mathsf{alg}}
\newcommand{\algOpAlg}[1]{\otimes #1}

\newcommand{\all}{\x{all}}
\newcommand{\greedy}{\x{greedy}}
\newcommand{\isFinal}[2]{\mathit{isFinal}_{#1}({#2})}
\newcommand{\isFinalPred}[1]{\mathit{isFinal}_{#1}}

