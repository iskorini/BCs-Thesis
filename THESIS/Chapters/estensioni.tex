\myChapter{Implementare Usage Control in FACPL}
\label{cap:usagecontrolfacpl}
FACPL, fino alla versione descritta nel capitolo~\ref{cap:facpl}, non aveva la possibilità
di essere sfruttato per \textit{Usage Control}.\\
Grazie a delle nuove strutture implementate insieme al mio collega Filippo Mameli, adesso è possibile
usare FACPL per \textit{Usage Control}, introducendo miglioramenti descritti in~\ref{sec:usage_control}.\\
La nuova funzionalità consiste nel prendere decisioni tenendo conto delle richieste già effettuate.
Introdurre questa nuova estensione ha richiesto del lavoro sulla libreria, in quanto è stato necessario aggiungere
nuove componenti e di conseguenza modificare il processo di valutazione di una policy.
Infine è stato necessario anche introdurre delle modifiche alla sintassi del linguaggio in modo da poterle sfruttare 
facilmente.

\section{Estensione del processo di Valutazione} % (fold)
\label{sec:estensione_del_processo_di_valutazione}
Il processo di valutazione è stato esteso per via delle modifiche introdotte. 
Rispetto al processo di valutazione standard, descritto in sezione~\ref{sec:valutazione_facpl}, sono state aggiunte
componenti al grafico, rendendolo così adatto allo \textit{Usage Control}, in particolare alla valutazione di 
richieste basate sul comportamento passato.
\MyFigure{evalStatus}{Nuovo processo di valutazione in FACPL}{1}
Come si nota in figura \ref{img:evalStatus} è stato aggiunto un componente alla struttura della valutazione.
Questo componente è lo \status \ (Stato), ovvero un semplice contenitore di un nuovo tipo di attributi.
I nuovi attributi vengono chiamati \statusattribute. Ovviamente quest'estensione non modifica il comportamento nel caso di assenza di stato, di conseguenza la valutazione rimane inalterata rispetto a quella descrita precedentemente, mentre viene modificata nel caso in cui lo stato sia presente in modo da gestire correttamente la presenza di esso.\\
Analizziamo quindi, a scopo esemplificativo, il secondo caso, ovvero quando lo stato è presente. Inizialmente viene definito il sistema, che ora ha quattro componenti principali:
\begin{itemize}
	\item[-]{Policy Repository (PR)}
	\item[-]{Policy Decision Point (PDP)}
	\item[-]{Policy Enforcement Point (PEP)}
	\item[-]{Status}
\end{itemize}
Fino al quarto step il comportamento è analogo a quello precedente, mentre cambia negli step successivi.\\
Al quinto step il \textit{PDP} non necessiterà solo dei normali attributi d'ambiente, ma necessiterà anche degli \statusattribute \ coinvolti nella richiesta effettuata. Il \textit{Context Handler} quindi non andrà solo a fare la ricerca all'interno dell'environment, ma andrà a cercare anche gli \statusattribute \ all'interno dello \status.\\
A questo punto, quando il PDP avrà tutte le informazioni necessarie si potrà passare alla vera e propria valutazione della richiesta che avviene come sempre.\\
Nel caso in cui viene restituto \texttt{Permit} o \texttt{Deny} è necessario fare l'enforcement della risposta del PDP. Questo processo differisce dal precedente poiché ora sono state implementate nuove azioni sullo stato che devono essere eseguite dal \textit{PEP} (Passo 13-16). Una volta effettuato l'enforcement viene restituta la decisione finale.\\
Prendendo il primo esempio citato in sezione \ref{sec:usage_control} la valutazione procederebbe in questo modo. Bob richiederà la lettura di un determinato file. Quindi la richiesta conterrà tre attributi, uno che indica il nome dell'utente che effettua la richiesta, il secondo che contiene il nome del file a cui si effettuerà l'accesso, e il terzo che conterrà il nome dell'azione da effettuare. La policy invece sarà strutturata come \textit{"Se il nome è Bob, il file è corretto e nessuno sta scrivendo o ci sono meno di due utenti che leggono, allora permetti, altrimenti nega" }.\\
Il PDP però ha bisogno di più attributi per valutare la richiesta, in quanto necessita anche di attributi esterni alla policy che riguardano il numero di utenti che stanno accedendo al file richiesto, questi attributi sono gli \statusattribute. Per la loro gestione sarà necessario utilizzare la funzionalità che riguarda l'Usage Control, il PDP richiederà al \textit{Context Handler} questi attributi, il quale andrà cercarli nello \status. Quest'ultimo li fornirà e verranno direttamente mandati al PDP per la valutazione della richiesta.\\
Se la richiesta avrà esito positivo, allora vuol dire che Bob avrà accesso al file, e quindi lo stato andrà aggiornato. La risposta del PDP a questo punto andrà al PEP per l'enforcement il quale avrà il compito di aggiornare lo stato. Sostanzialmente lo stato viene aggiornato semplicemente incrementando l'attributo riguardante il numero di lettori di un'unità.



% section estensione_del_processo_di_valutazione (end)

\section{Estensione Linguistica} % (fold)
\label{sec:estensione_linguistica}
Per implementare queste nuove funzionalità è stata modificata anche la grammatica di FACPL.
Nella grammatica estesa sono state aggiunte nuove regole di produzione e simboli terminali che 
codificano le nuove funzionalità.\\

\begin{table}[h]
\centering
\small
\caption{Sintassi di $FACPL_{PB}$} 
$
\begin{array}{@{\,}r@{\ \ }r@{\ }r@{\ \ }l@{\ }}

&&&\\[-.2cm]
{\textbf{Policy Authorisation Systems}} &
\mathit{PAS} & ::= & ( \,  \x{pep:} \, \mathit{EnfAlg}\ \ \x{pdp:}\, \mathit{PDP} \, \  \x({status:}\, \mathit{
[Attribute]^+})^*)
\\[.2cm]
{\textbf{Attribute}} &
\mathit{Attribute}
& ::= & (\mathit{Type} \ \mathit{Identifier} \ (= Value )^{?})
\\[.2cm]
{\textbf{Type}} &
\mathit{Type}
& ::= & \x{int} \ | \ \x{boolean} \ | \ \x{date} \ | \ \x{float}
\\[.2cm]
{\textbf{Enforcement algorithms}} &
\mathit{EnfAlg}
& ::= & \based \Sep \denyBiased \Sep \permitBiased 
\\[.2cm]
{\textbf{Policy Decision Points}} &
\mathit{PDP} & ::= & \pdpPol{\algNT\ }{\x{policies:} \, \mathit{Policy}^{+}}
\\[.2cm]
{\textbf{Combining algorithms}} &
\algNT & ::= & \permitOver \Sep \denyOver \Sep \denyUnless \Sep \permitUnless \\
&& \mid &
\firstApp \Sep \onlyOneApp \Sep \weakCon \Sep \strongCon 
\\[.2cm]
\textbf{fulfilment strategies} & \delta
& ::= & 
\greedy \Sep \all 
\\[.2cm]
{\textbf{Policies}} &
\mathit{Policy} & ::= &
\ruleOpt{\mathit{Effect}\ \ \x{target:} \, Expr\ \ \x{obl:} \, \mathit{Obligation}^{*} \, } \\
&& \mid &
\{ \algNT\ \ \x{target:} \, Expr\ \ \\
&&  &
\x{policies:} \, \mathit{Policy}^{+}  \ \ \x{obl:} \, \mathit{Obligation}^{*} \, \}
\\[.2cm]
{\textbf{Effects}} &
\mathit{Effect} & ::= & \permit \Sep \deny
\\[.2cm]
{\textbf{Obligations}} &
\mathit{Obligation} & ::= & [ \, \mathit{Effect} \ \ \mathit{ObType} \ \ \obl{Expr} \, ]
\\[.2cm]
{\textbf{PepAction}} & \mathit{PepAction} & ::= & \, \x{add(\mathit{Attribute}, int)} \Sep \x{flag(\mathit{Attribute}, boolean)} \\ 
&& & \Sep \x{sumDate(\mathit{Attribute}, date)} \Sep \x{div(\mathit{Attribute}, int)} \\
&& & \Sep \x{add(\mathit{Attribute}, float)} \ \Sep \x{mul(\mathit{Attribute}, float)} \\
&& & \Sep \x{mul(\mathit{Attribute}, int)} \ \Sep \x{div(\mathit{Attribute}, float)} \\
&& & \Sep \x{sub(\mathit{Attribute}, int)} \ \Sep \x{sub(\mathit{Attribute}, float)} \\
&& & \Sep \x{sumString(\mathit{Attribute}, string)} \\ 
&& & \Sep \x{setValue(\mathit{Attribute}, string)}\\
&& & \Sep \x{setDate(\mathit{Attribute}, date)}  
\\[.2cm]
{\textbf{Obligation Types}} &
\mathit{ObType} & ::= & M \Sep O
\\[.4cm]
\textbf{Expressions}&
\mathit{Expr} & ::= &
\mathit{Name} \Sep \mathit{Value}  \\
& & \mid &\x{and(\mathit{Expr}, \mathit{Expr})} \Sep \x{or(\mathit{Expr}, \mathit{Expr})} \Sep \x{not(\mathit{Expr})} \\
& & \mid &
 \x{equal(\mathit{Expr},\mathit{Expr})}  \Sep \x{in}(\mathit{Expr}, \mathit{Expr}) \\
& & \mid & \x{greater}\textrm{-}\x{than(\mathit{Expr},\mathit{Expr})} \Sep \x{add(\mathit{Expr} ,\mathit{Expr} )}\\ 
& & \mid & \x{subtract(\mathit{Expr} ,\mathit{Expr} )} \Sep \x{divide(\mathit{Expr} ,\mathit{Expr} )}\\
& & \mid & \x{multiply(\mathit{Expr} ,\mathit{Expr} )}  \Sep \x{less}\textrm{-}\x{than(\mathit{Expr}, \mathit{Expr})}\\
\\[.2cm]
%
\textbf{Attribute Names} & 
\mathit{Name} & ::= & \mathit{Identifier}/\mathit{Identifier} \ | \ \mathit{Status}/\mathit{Identifier}\\[.2cm]
%
\textbf{Literal Values} &
\mathit{Value} & ::= & \x{true} \mid \x{false} \mid \mathit{Double} \mid \mathit{String} \mid \mathit{Date}
\\[.4cm]
{\textbf{Requests}} &
\mathit{Request} & ::= & {\attribute{\mathit{Name}}{\mathit{Value}}}^{+}
\\[.1cm]
\end{array}
$
\label{tab:facpl_new_syntax}
\end{table}




Come è facilmente osservabile dalla consultazione della tabella~\ref{tab:facpl_new_syntax} le aggiunte rispetto alla tabella riporta in sezione~\ref{sec:facpl_syntax} sono state diverse, vediamo adesso quali sono.\\
La prima modifica che risulta evidente è nel PAS, ovvero nella definizione del sistema. L'aggiunta è stata lo \status, ovvero un contenitore di attributi.
Uno \status \ è della forma $$(status: Attribute^+)^?$$ questo significa che se lo \status \ è presente sarà formato da uno o più \textit{Attribute}.\\
Passiamo ora a descrivere \textit{Attribute} che è della forma $$(Type\ Identifier (= Value)^?)$$
questo tipo particolare di attribute, che è lo \statusattribute \ descritto in precedenza, è formato innanzitutto da un \textit{Type}, dopo il tipo è richiesta una generica stringa chiamata \textit{Identifier}, che sarà un generico nome da dare all'attributo, infine viene richiesto un \textit{Value}, ovvero un valore, che in questo caso è opzionale, all'atto pratico vuol dire che l'attributo di stato potrà essere inizializzato con un valore oppure potrà essere solamente definito, lasciando che il valore sia quello di default.\\
\textit{Type} è il tipo che avrà l'attributo di stato, e potrà essere \texttt{int, boolean, date o float}.\\
La regola \textit{PepAction} è stata modificata in modo tale che includesse nuove funzioni per operare matematicamente sugli attributi di stato.
Queste nuove funzioni sono:
\begin{itemize}
	\item \textit{add(Attribute, \texttt{int})}
	\item \textit{add(Attribute, \texttt{float})}
	\item \textit{div(Attribute, \texttt{int})}
	\item \textit{div(Attribute, \texttt{float})}
	\item \textit{sub(Attribute, \texttt{int})}
	\item \textit{sub(Attribute, \texttt{float})}
	\item \textit{mul(Attribute, \texttt{int})}
	\item \textit{mul(Attribute, \texttt{float})}
	\item \textit{flag(Attribute, \texttt{boolean})}
	\item \textit{sumDate(Attribute, \texttt{date})} 
	\item \textit{sumString(Attribute, \texttt{string})}
	\item \textit{setValue(Attribute, \texttt{string})}
	\item \textit{setDate(Attribute, \texttt{date})}
\end{itemize}
Infine l'ultima regola di produzione modificata è stata quella riguardante \textit{Attribute Names}, in questo caso è stata semplicemente aggiunto, a fianco di \textit{Identifier/Identifier}, una nuova produzione \textit{Status/Identifier}. Questa nuova produzione serve semplicemente per permettere il confronto tra attributi di stato attraverso le già esistenti \textit{Expression}.
La sintassi delle risposte è rimasta invariata.
Vediamo ora un esempio di questa nuova sintassi, prenderemo spunto da un caso già trattato in precedenza nella sezione~\ref{sec:estensione_del_processo_di_valutazione}.
%SOSTITUIRE CON LISTINGS PER FACPL
\lstinputlisting[language = FACPL, caption = {Esempio per la sintassi}\label{lst:esempio_sintassi}]{./Source/first_example_facpl}
In questo esempio (Codice \ref{lst:esempio_sintassi}) si può vedere come nel PAS è stato definito uno stato, con al suo interno uno solo attributo inizializzato con valore 0.
Successivamente si può notare nella \textit{Rule} che viene fatto un controllo sul valore di quest'attributo.
Infine nella \textit{Obligation} si può notare come viene aggiornato lo stato dell'attributo in base al risultato della valutazione della \textit{Rule}.
% section estensione_linguistica (end)

\section{Semantica} % (fold)
\label{sec:semantica}
Sostanzialmente la semantica di FACPL rimane molto simile a quella descritta in sezione \ref{sec:semantica_originale}, quindi verranno di seguito descritte in modo informale solo le novità introdotte.\\
La prima di queste riguarda la valutazione delle richieste dal PDP. Il PDP ora non si deve più basare solo su richieste totalmente scollegate l'una dall'altra, e quindi è stato introdotto il concetto di \status.
Lo stato serve per l'appunto per memorizzare il comportamento passato del sistema, e lo fa introducendo una nuova serie di attributi chiamati \statusattribute.\\
Nel linguaggio questo nuovo tipo di attributi viene considerato al pari di normali attributi, quindi si ha la possibilità di effettuare tutte le operazioni di confronto tra di essi, ma in più si deve avere la possibilità di modificarli e memorizzarli in modo da poterli sfruttare per \textit{Usage Control}.\\
Gli \statusattribute \ vengono inizialmente definiti nel PAS, ed avranno un tipo, un identificatore ed eventualmente un valore. Per esempio uno \statusattribute definito in questo modo \textit{(boolean isWriting = false)} codifica un attributo di stato di tipo booleano chiamato "isWriting" ed inizializzato a \texttt{false}.\\
Come detto prima bisogna avere la possibilità di modificare questi nuovi attributi, e per questo sono state aggiunte delle \textit{Pep Action}, ovvero delle azioni eseguite dal PEP in seguito alla valutazione di \textit{Obligations}.
\subsection*{\textit{add(Attribute, int)} e \textit{add(Attribute, float)}} % (fold)
\label{ssub:opadd}
Queste due \textit{Pep Actions} effettuano un'azione su un attributo di stato. Questa \textit{Pep Action} permette, in seguito alla valutazione di una \textit{Obligations}, l'aggiunta di un valore numerico ad uno \statusattribute \ di tipo \texttt{float} o \texttt{int}. Effettuare questa operazione su un tipo non permesso porterà ad un errore in fase di valutazione.
\begin{verbatim}
 obl:
     [permit M add(counter, 2)]
\end{verbatim}
Per esempio l'esecuzione di questa \textit{Obligation} su un'attributo, chiamato \textit{counter}, inizializzato a 0, porterà l'attributo al valore 2.
Mentre l'esecuzione di questo
\begin{verbatim}
 obl:
     [permit M add(counter, "foo")]
\end{verbatim}
porterà ad un errore per incoerenza sul tipo.
% subsection add (end)
\subsection*{\textit{sub(Attribute, int)} e \textit{sub(Attribute, float)}} % (fold)
\label{ssub:opsub}
Questo insieme \textit{Pep Actions} è semplicemente il duale della precedente, ciò vuol dire che invece che effettuare la somma di un valore numerico ad uno \statusattribute \ farà una sottrazione.\\
Quindi, riprendendendo l'esempio citato per la add:
\begin{verbatim}
 obl:
     [permit M sub(counter, 2)]
\end{verbatim}
l'esecuzione corretta di questa \textit{Obligation} porterà nuovamente l'attributo ad il suo valore originale, ovvero 0. Ovviamente riguardo gli errori vale lo stesso discorso fatto in precedenza.
% subsection subsection_name (end)

\subsection*{\textit{div(Attribute, int)} e \textit{div(Attribute, float)}} % (fold)
\label{ssub:opdiv}
Le \textit{Pep Actions} di questa categoria effettuano una divisione tra un attributo ed un numero passatogli come parametro.
\begin{verbatim}
 obl:
     [permit M div(number, 2)]
\end{verbatim}
Presupponendo che il valore dell'attributo \textit{number} sia inizialmente 6 l'esecuzione di questa operazione lo porterà a $6/2$, ovvero 3.
% subsection div(attribute, int)_e_div(attribute, float) (end)

\subsection*{\textit{mul(Attribute, int)} e \textit{mul(Attribute, float)}} % (fold)
\label{ssub:opdiv}
Quest'azione, come facilmente intuibile dal nome effettua la moltiplicazione.
Considerando l'esempio fatto per la div
\begin{verbatim}
 obl:
     [permit M mul(number, 3)]
\end{verbatim}
Visto che il valore dell'attributo, modificato in precedenza dalla div, è 3 l'operazione effettuata sarà $3*3$ che darà come risultato 9.


% subsection div(attribute, int)_e_div(attribute, float) (end)

\subsection*{flag(Attribute, boolean)} % (fold)
\label{ssub:opflag}
Quest'operazione, definita solo sul tipo \texttt{boolean} modifica il valore di uno \statusattribute \ di tipo booleano.
\begin{verbatim}
 obl:
     [permit M flag(isFoo, true)]
\end{verbatim}
L'esecuzione con successo di questa \textit{Obligation} porterà l'attributo \textit{flag} ad avere un valore \texttt{true}.
% subsection subsection_name (end)

\subsection*{sumDate(Attribute, date)}
\label{ssub:opdate}

Anche questa azione, come quella di prima è definita su un solo tipo, ma questa volta il tipo è \texttt{Date}. Il tipo \texttt{Date} può essere espresso con in tre forme diverse
\begin{itemize}
\item{\texttt{HH:mm:ss}}
\item{\texttt{yyyy/MM/dd}}
\item{\texttt{yyyy/MM/dd-HH:mm:ss}}
\end{itemize}
Per fare un altro esempio consideriamo uno \statusattribute \ di tipo \texttt{date} inizializzato al giorno \texttt{2016/04/20} di nome \textit{foo}.
\begin{verbatim}
	 obl:
     [permit M sumDate(foo, 24:00:00)]
\end{verbatim}
La corretta esecuzione di questa \textit{Obligation} porterà il valoer di \textit{foo} al giorno successivo, ovvero il \texttt{21 Aprile 2016}.

\subsection*{sumString(Attribute, string)}
\label{ssub:opsumstring}

L'operazione ora descritta coinvolge il tipo \texttt{string}. Passandogli un attributo
di tipo \texttt{String} ed una stringa effettuerà la concatenazione.
\begin{verbatim}
	 obl:
     [permit M sumString(Pablo, " Neruda")]
\end{verbatim}
Se \texttt{Pablo} avesse il valore \textit{"Pablo"}, il risultato di questa operazione farebbe cambiare valore all'attributo in \textit{"Pablo Neruda"}.

\subsection*{setValue(Attribute, string)}
\label{ssub:opsetvalue}

Quest'operazione è molto simile alla precedente, ma invece che concatenare le stringhe sostituisce il valore presente dentro l'attributo con quello passatogli come parametro.
\begin{verbatim}
	 obl:
     [permit M setValue(Pablo, "Aghiò Aghiò")]
\end{verbatim}
Quindi l'esecuzione corretta di quest'azione porterà l'attributo che precedentemente aveva il valore 
\textit{"Pablo Neruda"} ad avere il valore \textit{"Aghiò Aghiò"}.


\subsection*{setDate(Attribute, date)}
\label{ssub:opsetvalue}

Quest'operazione è uguale alla precedente, ma invece che operare sul tipo stringa opera sul tipo Data.
\begin{verbatim}
	 obl:
     [permit M setDate(foo, 2016/12/25)]
\end{verbatim}
L'esecuzione di questa \textit{Obligation} modificherà il valore di \textit{foo} al 25 Dicembre 2016.
% section semantica (end)


\section{Esempi} % (fold)
\label{sec:esempi}

Queste nuove funzionalità introdotte servono allo scopo descritto in sezione~\ref{sec:usage_control}, ovvero l'implementazione di un nuovo modello chiamato \textit{Usage Control}.
Mostreremo ora l'implementazione in FACPL dei due esempi trattati in sezione~\ref{sec:usage_control}. 
\subsection{Primo esempio} % (fold)
\label{ssub:primo_esempio}
\lstinputlisting[language = FACPL, caption = {Primo esempio}\label{lst:PrimoEsempio_FACPL}]{./Source/EsempioReadWrite_facpl.fpl}
Per semplicità, in questo codice, vengono mostrate le policy di accesso per un solo file.
L'esempio citato in precedenza metteva una regola sull'accesso ai file, ovvero permetteva un massimo di due persone in contemporanea che potevano effettuare l'accesso in lettura oppure un massimo di una persona che poteva ottenere l'accesso in scrittura.\\
Tutte le policy del codice~\ref{lst:PrimoEsempio_FACPL} sono chiuse in un \textit{PolicySet}. Il target del PolicySet \texttt{ReadWrite\_Policy} verifica che le richieste provengano da utenti che hanno nome \textit{Alice} o \textit{Bob}, in caso contrario il responso sarà \texttt{Not Applicable}.\\
Successivamente sono state create quattro policy per gestire le quattro operazioni possibili, ovvero \texttt{read}, \texttt{write}, \texttt{stopRead} ed infine \texttt{stopWrite}.\\
Prendiamo in considerazione la prima policy, quella per la \texttt{write}. Come prima è presente un target, che richiede questa volta due diverse condizioni, la prima riguarda l'id del file richiesto, la seconda invece richiede che l'azione sia \texttt{write}. Le parti interessanti di questa policy sono due, la prima riguarda la \textit{Rule}, la seconda la \textit{Obligation}.\\
La \textit{Rule} restituisce \texttt{permit} se le due condizioni dell'equal sono vere, come si può facilmente notare l'operazione di confronto non viene fatta tra una stringa ed un normale attributo, ma tra una stringa ed uno \statusattribute.\\
L'ultima cosa da notare è l'unica \textit{Obligation} presente per questa policy. Questo tipo particolare di \textit{Obligation}, ha sempre al suo interno un'azione che verrà eseguita dal PEP, ma questa volta l'azione andrà a modificare lo stato del sistema, mettendo il valore \texttt{true} all'attributo \textit{isWriting}.\\
Prendiamo ora una serie di richieste.
\lstinputlisting[language = FACPL, caption = {Richieste del primo esempio}\label{lst:PrimoEsempioRichieste_FACPL}]{./Source/EsempioReadWrite_facpl_richieste.fpl}
L'output di queste richieste sarà il sguente
\begin{verbnobox}[\small]
 Request1:
	  PDP Decision=
 	  Decision: PERMIT Obligations: PERMIT M AddStatus([INT/counterReadFile1/0, 1])
	  PEP Decision= PERMIT
 	
 Request2:
	  PDP Decision=
 	  Decision: DENY Obligations: 
	  PEP Decision= DENY
 	
 Request3:
	  PDP Decision=
 	  Decision: PERMIT Obligations: PERMIT M AddStatus([INT/counterReadFile1/1, 1])
	  PEP Decision= PERMIT
 	
 Request4:
	  PDP Decision=
 	  Decision: PERMIT Obligations: SubStatus([INT/counterReadFile1/2, 1])
	  PEP Decision= PERMIT
 	
 Request5
	  PDP Decision=
 	  Decision: PERMIT Obligations: PERMIT M SubStatus([INT/counterReadFile1/1, 1])
	  PEP Decision= PERMIT
 	
 Request6:
	  PDP Decision=
 	  Decision: PERMIT Obligations: PERMIT M FlagStatus([BOOLEAN/isWriting/false, true])
	  PEP Decision= PERMIT
 \end{verbnobox}
Analizziamo ora il motivo di queste decisioni. Nella prima richiesta ovviamente nessuno sta leggendo o scrivendo, quindi viene tranquillamente restituito \permit, visto che è presente una \textit{obligation} lo stato verrà aggiornato, sommando un'unità al contatore di letture.
Alla seconda richiesta l'utente richiede la scrittura, che gli viene negata perché c'è già qualcuno che sta leggendo, però lo stesso utente effettua un'altra richiesta, questa volta in lettura, che gli viene concessa.\\
La quarta e la quinta richiesta vengono fatte per avvisare il sistema che la lettura è terminata, ovviamente la risposta è \permit, e la \textit{obligation} corrispondente decrementerà il contatore.
La sesta ed ultima richiesta è una scrittura, che questa volta viene permessa, poiché nessuno sta scrivendo o leggendo.


% subsection primo_esempio (end)

\subsection{Secondo Esempio} % (fold)
\label{sub:secondo_esempio}

% subsection secondo_esempio (end)

% section esempi (end)
