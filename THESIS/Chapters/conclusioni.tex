\myChapter{Conclusioni}
\label{cap:conclusioni}
Durante questa tesi è stata affrontato il lavoro di implementazione di \textit{Usage Control} in \ac{FACPL}.
Come primo compito ci siamo occupati di analizzare i principali modelli dedicati all'\textit{Access Control} e successivamente è seguita una fase di approfondimento sul modello \textit{Usage Control} proposto da Sandhu e Park.  \par
Il lavoro è seguito con una disamina sul linguaggio \ac{FACPL} in modo da comprendere al meglio la sintassi, la sematica e soprattutto il processo di valutazione così da avere un background e sapere come, e dove, intervenire per implementare del concetto di \status \ e di tutte le cose che conseguentemente ne derivano da esso, come nuovi attributi, nuove obligation e funzioni per operare su attributi. \par
Nella sintassi estesa sono state aggiunte nuove regole di produzione e ne sono state modificate alcune. Quelle modificate includono la definizione del sistema, mentre quelle aggiunte riguardano nuove funzioni, ed un nuovo tipo di attributo, chiamato \statusattribute.
Successivamente è stato svolto del lavoro sulla libreria Java per implementare queste nuove caratteristiche. \par
Nel capitolo \ref{cap:estensione_libreria} viene descritta l'implementazione delle nuove caratteristiche del linguaggio in Java. All'inizio del Capitolo viene descritta l'implementazione dello \status \ e di conseguenza degli \statusattribute. Successivamente sono è stata necessaria  l'implementazione  delle funzioni per la modifica di questi attributi in modo da poter cambiare lo stato con l'avanzare delle valutazioni delle richieste. L'estensione ha coinvolto anche il \ac{PEP}in quanto deve valutare un nuovo tipo di Obligation, ovvero le Obligation Status.
La differenza con le Obligation normali risiede nel fatto che le Obligation Status possono eseguire le funzioni per la modifica dello stato mostrate all'inizio del capitolo.
Durante la valutazione di una richiesta devono essere valutati anche gli attributi di stato. Permettere la valutazione di quest'ultimi da parte delle funzioni già esistenti è stato abbastanza semplice, poiché è bastato estendere la classe che implementa il contesto in modo tale che facesse la ricerca degli attributi anche all'interno dello stato.

\section{Sviluppi futuri}
\label{sec:futuro}

Gli esempi mostrati durante questa tesi sono basilari, e fondamentalmente sono stati scritti con il solo scopo di provare il funzionamento delle nuove funzionalità di \ac{FACPL}. Potrebbe risultare interessante applicare \ac{FACPL} a casi di studio reali, in modo da poterne verificare le potenzialità sul campo.\par
Durante lo sviluppo sono stati implementati solo alcuni tipi di dato e relative funzioni su di essi, in futuro sarebbe facilmente possibile implementarne di nuovi in quanto la libreria è stata
progettata per favorirne la rapida e semplice estendibilità. Per esempio potrebbe essere stimolante l'implementazione di un tipo nuovo come le Liste e funzioni come ricerca all'interno di esse, aggiunta ed eliminazione di elementi o il conteggio del numero di elementi.