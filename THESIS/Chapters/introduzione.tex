\myChapter{Introduzione}
FACPL è un linguaggio basato su XACML, supportato da una libreria
scritta in Java. Rispetto a XACML la sintassi di FACPL è molto più semplice
e concisa, quindi permette di formalizzare in modo facile e rapido
politiche di \textit{access control}. Tuttavia, FACPL non gode di caratteristiche
per eseguire richieste ed ottenere risposte valutando
anche il comportamento passato, quindi l’obiettivo di questa tesi è stato
quello di estenderlo in modo da implementare nuove e fondamentali funzionalità
che permettono di sfruttare questo nuovo modellom chiamato \textit{Usage Control}. \\ \par


La questione sul controllo agli accessi è un problema rilevante al
quale si è provato a trovare una soluzione negli ultimi anni. In particolare,
a partire dal 1970, sono stati proposti molti modelli (come ACL,
RBAC, ABAC e PBAC), ognuno dei quali ha i suoi pro e contro.
L'\textit{ACL}, o \textit{Access Control List}, è un modello basato su liste di accesso associate
ad ogni risorsa che definiscono le regole di accesso; è un modello molto
semplice da implementare, ma quando i dati sono eccessivi diventa
poco efficiente. Successivamente è stato introdotto un modello basato su
ruoli, il quale associa a degli insiemi di risorse delle mansioni, ovvero dei gruppi
di utenti suddivisi secondo una caratteristica comune. Un punto debole
di questo modello è la mancata specificità riguardante la definizione
delle regole che rende difficile assegnare permessi particolari a singole
risorse. A seguire è stato utilizzato un modello basato su attributi dove
le decisioni vengono prese valutando caratteristiche del richiedente,
della risorsa e dell’ambiente in cui è stata fatta la richiesta. Quest’ultimo
modello tuttavia non offre una buona scalabilità. Per porre
rimedio a questo difetto è stato introdotto un ulteriore modello,
che oggi risulta essere uno dei più utilizzati, basato su politiche,
ovvero un insieme di regole su attributi. Recentemente, Sandhu e Park
hanno introdotto un’estensione di \textit{Access Control} chiamata \textit{Usage Control}
che permette di prendere decisioni durante
l’accesso e di basarsi sul comportamento passato in fase di valutazione
di una richiesta. Questa caratteristica lo rende molto adatto ad ambienti come il Web,
il Cloud o in generale legati in qualche modo alla rete.\\ \par


La tesi, dopo questa breve introduzione, è organizzata in questo modo: 
\begin{itemize}
\item Nel capitolo~\ref{cap:accessControl} vengono
analizzati in maggior dettaglio tutti i modelli di \textit{Access Control} accennati
in precedenza ed inoltre viene dedicata una sezione a \textit{Usage Control}
dove vengono anche introdotti due esempi che verranno portati avanti
durante gli altri capitoli.
\item Nel capitolo~\ref{cap:facpl} è descritto FACPL allo stato
precedente alla stesura di questo documento, ovvero quando ancora non
era possibile esprimere politiche di accesso basate sul comportamento
passato. Alla fine del capitolo verrà proposto un esempio e sarà
spiegato perché non è possibile prendere decisioni di \textit{Usage Control}.
\item Nel capitolo~\ref{cap:usagecontrolfacpl} viene trattata l’estensione di FACPL ad un livello sintattico
e semantico, spiegando quali nuove componenti sono state introdotte ed
analizzandone il loro significato ed utilizzo in un contesto reale.
Usando le nuove estensioni, a fine capitolo sono stati riproposti in FACPL gli esempi citati in
ezione~\ref{sec:usage_control}.
\item Nel quinto e ultimo capitolo, infine, sono trattati gli argomenti
visti in quello precedente ma dal punto di vista dell’estensione della
libreria Java.
\end{itemize}
