\myChapter{Introduzione}
I sistemi informatici moderni sono sempre più interconnessi tra di loro, quindi
impedire accessi indesiderati è diventato un argomento di studio molto rilevante negli ultimi decenni.
A partire dal 1970 sono stati proposti molti sistemi di controllo agli accessi, ognuno con i suoi pro ed i suoi contro.\par
\textit{Usage Control} è un nuovo approccio sul controllo degli accessi
introdotto recentemente da Sandhu e Park \cite{ucon}.
Quest ultimo permette di ottenere decisioni durante
l’accesso e di basarsi sul comportamento passato in fase di valutazione
di una richiesta. Questa caratteristica lo rende molto adatto ad ambienti come il Web,
il Cloud o in generale legati in qualche modo alla rete. \par
L'obiettivo di questa tesi è estendere \ac{FACPL} in modo tale da poter prendere decisioni 
di Usage Control.
\ac{FACPL} è un linguaggio basato su \ac{XACML} ed è supportato da una libreria
scritta in Java. Rispetto a \ac{XACML} la sintassi di \ac{FACPL} è molto più semplice
e concisa, quindi permette di formalizzare in modo facile e rapido
politiche di \textit{Access Control}. 
Tuttavia, \ac{FACPL} non gode di caratteristiche
per eseguire richieste ed ottenere risposte valutando il comportamento passato. Questo è dovuto alla mancanza di un componente che permette di tenere traccia del comportamento precedente, che può essere assimilabile ad uno stato.\par
L'estensione di \ac{FACPL} è stata divisa fondamentalmente in due parti. 
Nella prima parte del lavoro è stata elaborata una nuova grammatica, una nuova semantica ed un processo di valutazione delle policy esteso che includesse anche lo stato.
La seconda parte del lavoro, a differenza della prima, è stata più concreta poiché in questa fase è stata estesa la libreria Java in modo da dare un supporto alle estensioni pensate precedentemente. \par



La tesi, dopo questa breve introduzione, è organizzata in questo modo: 
\begin{itemize}
\item Nel Capitolo~\ref{cap:accessControl} vengono
presentati in maggior dettaglio tutti i modelli di \textit{Access Control} ed inoltre viene dedicata una sezione a \textit{Usage Control}
dove vengono anche introdotti due esempi che verranno portati avanti
durante gli altri capitoli.
\item Nel Capitolo~\ref{cap:facpl} è descritto \ac{FACPL} e vengono riportati alcuni esempi che mostrano le problematiche per cui non è possibile farne un uso a livello di Usage Control.
\item Nel Capitolo~\ref{cap:usagecontrolfacpl} viene trattata l’estensione di \ac{FACPL} ad un livello sintattico
e semantico, spiegando quali nuove componenti sono state introdotte ed
analizzandone il loro significato ed utilizzo attraverso dei casi di studio.
\item Nel Capitolo~\ref{cap:usagecontrolfacpl} sono trattati gli argomenti
visti in quello precedente ma dal punto di vista dell’estensione della
libreria Java.
\item Nel Capitolo~\ref{cap:conclusioni} viene riassunto tutto il lavoro svolto e sono proposte idee per gli sviluppi futuri.

\item In Appendice~\ref{cap:appendiceA} viene proposto il codice completo.
\end{itemize}
