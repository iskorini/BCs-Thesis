\myChapter{Introduzione}


I sistemi informatici si sono diffusi molto rapidamente, e grazie all'avvento di nuove tecnologie, come la rete, la condivisione di dati e risorse è diventata alla portata di tutti. Proteggere queste risorse da accessi 
indesiderati è diventato molto importante, motivo per cui negli ultimi decenni la questione 
del \textit{Access Control} è diventata sempre più rilevante.
In particolare,
a partire dal 1970, sono stati proposti vari modelli (come \ac{ACL},
\ac{RBAC}, \ac{ABAC} e \ac{PBAC}), ognuno dei quali ha i suoi pro e contro.


Il primo modello proposto è \ac{ACL} ed ha come obiettivo il controllo delle risorse di un sistema operativo. Questo modello si basa su liste di accesso associate
ad ogni risorsa che definiscono le regole di accesso; è un modello molto
semplice da implementare, ma quando i dati sono eccessivi diventa
poco efficiente. 
Successivamente è stato introdotto un modello basato su
ruoli, ovvero dei gruppi
di utenti suddivisi secondo caratteristiche comuni, chiamato \ac{RBAC}, il quale associa a degli insiemi di risorse dei diritti di accesso. Un punto debole
di questo modello è l'assenza di costrutti che permettono la definizione
delle regole che rende difficile assegnare permessi particolari a singole
risorse. A seguire è stato utilizzato un modello basato su attributi chiamato \ac{ABAC}, dove
le decisioni vengono prese valutando caratteristiche del richiedente,
della risorsa e dell’ambiente in cui è stata fatta la richiesta. Quest’ultimo
modello tuttavia non offre una buona scalabilità. Per porre
rimedio a questo difetto è stato introdotto un ulteriore modello, che si chiama \ac{PBAC}
il quale risulta essere uno dei più utilizzati ed è basato su politiche, 
ovvero un insieme di regole su attributi. 

Recentemente, Sandhu e Park \cite{ucon}
hanno introdotto \textit{Usage Control}
che permette di prendere decisioni durante
l’accesso e di basarsi sul comportamento passato in fase di valutazione
di una richiesta. Questa caratteristica lo rende molto adatto ad ambienti come il Web,
il Cloud o in generale legati in qualche modo alla rete. \par

\ac{FACPL} è un linguaggio basato su \ac{XACML}, supportato da una libreria
scritta in Java. Rispetto a \ac{XACML} la sintassi di \ac{FACPL} è molto più semplice
e concisa, quindi permette di formalizzare in modo facile e rapido
politiche di \textit{access control}. Tuttavia, \ac{FACPL} non gode di caratteristiche
per eseguire richieste ed ottenere risposte valutando
anche il comportamento passato, quindi l’obiettivo di questa tesi è stato
quello di estenderlo in modo da implementare nuove e fondamentali funzionalità
che permettono di sfruttare caratteristiche tipiche di \textit{Usage Control}.  \par

La tesi, dopo questa breve introduzione, è organizzata in questo modo: 
\begin{itemize}
\item Nel Capitolo~\ref{cap:accessControl} vengono
presentati in maggior dettaglio tutti i modelli di \textit{Access Control} ed inoltre viene dedicata una sezione a \textit{Usage Control}
dove vengono anche introdotti due esempi che verranno portati avanti
durante gli altri capitoli.
\item Nel Capitolo~\ref{cap:facpl} è descritto \ac{FACPL} e vengono riportati alcuni esempi che mostrano le problematiche per cui non è possibile farne un uso a livello di Usage Control.
\item Nel Capitolo~\ref{cap:usagecontrolfacpl} viene trattata l’estensione di \ac{FACPL} ad un livello sintattico
e semantico, spiegando quali nuove componenti sono state introdotte ed
analizzandone il loro significato ed utilizzo attraverso dei casi di studio.
\item Nel Capitolo~\ref{cap:usagecontrolfacpl} sono trattati gli argomenti
visti in quello precedente ma dal punto di vista dell’estensione della
libreria Java.
\item Nel Capitolo~\ref{cap:conclusioni} viene riassunto tutto il lavoro svolto e sono proposte idee per gli sviluppi futuri.

\item In Appendice~\ref{cap:appendiceA} viene proposto il codice completo.
\end{itemize}
