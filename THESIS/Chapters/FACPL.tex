\myChapter{Formal Access Control Policy Language}
\label{cap:facpl}
Negli anni molti linguaggi sono stati proposti per definire policy di access control. Uno di questi è \acf{XACML} di OASIS, e la sua prima versione risale al $2003$. Questo linguaggio ha una sintassi basata su \ac{XML} e fornisce caratteristiche avanzate per l'\textit{Access Control}. Il problema fondamentale di \ac{XACML} è che non ha una sintassi facile da leggere e da scrivere. \par
\acf{FACPL} è un linguaggio per formalizzare policy di access control supportato da una solida libreria Java ed utilizzabile attraverso un plugin per il famoso ambiente di sviluppo Eclipse.
L'obiettivo di \ac{FACPL} è definire una sintassi alternativa per \ac{XACML} in modo da renderlo più agevole da usare.
\ac{FACPL} quindi è parzialmente inspirato a \ac{XACML}, ma oltre ad introdurre una nuova sintassi ridefinisce alcuni aspetti aggiungendo nuove caratteristiche. Il suo scopo però non è sostituire \ac{XACML}, ma fornire un linguaggio compatto ed espressivo per facilitare le tecniche di analisi attraverso tool specifici.\par
In Sezione~\ref{sec:tool} sono descritti i tool necessari per l'utilizzo di 
\ac{FACPL}.
Nella Sezione~\ref{sec:valutazione_facpl} è effettuata una disamina sul processo di valutazione di \ac{FACPL}, nella quale sono presentati i componenti principali e la descrizione dell'interazione tra di essi.
Nelle Sezioni~\ref{sec:facpl_syntax} e ~\ref{sec:semantica_originale} viene analizzata rispettivamente la sintassi e la semantica di \ac{FACPL}.
In Sezione~\ref{sec:esempio_di_politica_con_facpl} è proposto un esempio di politica con \ac{FACPL} ed inoltre è spiegato in maggior dettaglio perché non è possibile prendere decisioni basate sul comportamento passato.





\section{Processo di valutazione}
\label{sec:valutazione_facpl}


\MyFig{FACPL_EVALUATION.jpg}{Il processo di valutazione di FACPL}{1}{t}
In figura \ref{fig:FACPL_EVALUATION.jpg} è mostrato il processo di valutazione delle policy definite in FACPL.
I componenti principali sono tre:
\begin{itemize}
\item{\acf{PR}}
\item{\acf{PDP}}
\item{\acf{PEP}}
\end{itemize}
Le policy sono memorizzate nel \ac{PR}, il quale le rende disponibili al \ac{PDP} che deciderà, successivamente, se garantire l'accesso o meno (Primo step).
Nello step 2, quando il \ac{PEP} riceve una richiesta, le credenziali di quest'ultima vengono codificate in una sequenza di attributi (ogni attributo è una coppia stringa valore) che, nello step 3, andranno a loro volta a formare una \ac{FACPL} Request.
Al quarto step il \textit{context handler} aggiungerà attributi di ambiente (per esempio l'ora di ricezione della richiesta) e manderà la richiesta al \ac{PDP}.
A questo punto il \ac{PDP}, tra il quinto e l'ottavo step, valuterà la richiesta e fornirà un risultato, il quale può eventualmente contenere delle \textit{obligations}.
La decisione del \ac{PDP} può essere di quattro tipi:
\begin{itemize}
\item \texttt{permit}
\item \texttt{deny}
\item \texttt{not-applicable} 
\item \texttt{indeterminate}.
\end{itemize}
Il significato delle prime due decisioni è facilmente intuibile, mentre per le ultime due vuol dire che c'è stato un errore durante la valutazione.
Gli errori possono essere di diverso tipo, e vengono gestiti attraverso algoritmi che combinano le decisioni delle varie policy per ottenere un risultato finale.
Le \textit{obligations} sono azioni, eseguite dal \ac{PEP}, correlate al sistema di controllo degli accessi. Queste azioni possono essere di svariati tipi, come per esempio generare un file di log, o mandare una mail.
Allo step 13, sulla base del risultato delle \textit{obligations}, il ac{PEP} esegue un processo chiamato \textit{Enforcement} il quale restituirà un'altra decisione.
Quest'ultima decisione corrisponde alla decisione finale del sistema e può differire da quella del PDP.


\section{Sintassi}
\label{sec:facpl_syntax}



\begin{table}[]
\footnotesize
\caption{Sintassi di FACPL}
\hrule
$
\begin{array}{@{\,}r@{\ \ }r@{\ }r@{\ \ }l@{\ }}
&&&\\[-.2cm]
{\textbf{Policy Authorisation Systems}} &
\mathit{PAS} & ::= & ( \,  \x{pep:} \, \mathit{EnfAlg}\ \ \x{pdp:}\, \mathit{PDP} \, )
\\[.2cm]
{\textbf{Enforcement algorithms}} &
\mathit{EnfAlg}
& ::= & \based \Sep \denyBiased \Sep \permitBiased 
\\[.2cm]
{\textbf{Policy Decision Points}} &
\mathit{PDP} & ::= & \pdpPol{\algNT\ }{\x{policies:} \, \mathit{Policy}^{+}}
\\[.2cm]
{\textbf{Combining algorithms}} &
\algNT & ::= & \permitOver \Sep \denyOver \Sep \denyUnless \Sep \permitUnless \\
&& \mid &
\firstApp \Sep \onlyOneApp \Sep \weakCon \Sep \strongCon 
\\[.2cm]
{\textbf{Policies}} &
\mathit{Policy} & ::= &
\ruleOpt{\mathit{Effect}\ \ \x{target:} \, Expr\ \ \x{obl:} \, \mathit{Obligation}^{*} \, } \\
&& \mid &
\{ \algNT\ \ \x{target:} \, Expr\ \ 
\x{policies:} \, \mathit{Policy}^{+} \ \ \x{obl:} \, \mathit{Obligation}^{*} \, \}
\\[.2cm]
{\textbf{Effects}} &
\mathit{Effect} & ::= & \permit \Sep \deny
\\[.2cm]
{\textbf{Obligations}} &
\mathit{Obligation} & ::= & [ \, \mathit{Effect} \ \ \mathit{ObType} \ \ \obl{Expr} \, ]
\\[.2cm]
{\textbf{Obligation Types}} &
\mathit{ObType} & ::= & M \Sep O
\\[.4cm]
\textbf{Expressions}&
\mathit{Expr} & ::= &
\mathit{Name} \Sep \mathit{Value}  \\
& & \mid &\x{and(\mathit{Expr}, \mathit{Expr})} \Sep \x{or(\mathit{Expr}, \mathit{Expr})} \Sep \x{not(\mathit{Expr})} \\
& & \mid &
 \x{equal(\mathit{Expr},\mathit{Expr})}  \Sep \x{in}(\mathit{Expr}, \mathit{Expr}) \\
& & \mid & \x{greater}\textrm{-}\x{than(\mathit{Expr},\mathit{Expr})} \Sep \x{add(\mathit{Expr} ,\mathit{Expr} )}\\ 
& & \mid & \x{subtract(\mathit{Expr} ,\mathit{Expr} )} \Sep \x{divide(\mathit{Expr} ,\mathit{Expr} )}\\
& & \mid & \x{multiply(\mathit{Expr} ,\mathit{Expr} )} \\[.2cm]
%
\textbf{Attribute Names} & 
\mathit{Name} & ::= & \mathit{Identifier}/\mathit{Identifier} \\[.2cm]
%
\textbf{Literal Values} &
\mathit{Value} & ::= & \x{true} \mid \x{false} \mid \mathit{Double} \mid \mathit{String} \mid \mathit{Date}
\\[.4cm]
{\textbf{Requests}} &
\mathit{Request} & ::= & {\attribute{\mathit{Name}}{\mathit{Value}}}^{+}
\\[.1cm]
\end{array}
$
\hrule
\label{tab:facpl_syntax}
\end{table}

\begin{table}[]
\footnotesize

\caption{Sintassi ausiliaria per le risposte}
\hrule
$
\begin{array}{@{\ }r@{\ \ \ \ }r@{\ }r@{\ \ }l@{\ }}

&&&\\[-.2cm]
{\textbf{PDP \ Responses }} &
\mathit{PDPResponse} & ::= & \langle \,\mathit{Decision} \ \ \ \mathit{FObligation}^* \rangle
\\[.2cm]
{\textbf{Decisions}} &
\mathit{Decision} & ::= & \permit \Sep \deny \Sep \notApp \Sep \indet
\\[.2cm]
{\textbf{Fulfilled obligations}} &
\mathit{FObligation} & ::= &  [ \, \mathit{ObType} \ \ \obl{\mathit{Value}} \, ]\\[.1cm]

\end{array}
$\\
\hrule
\label{tab:facpl_context_syntax}
\end{table}


La sintassi di \ac{FACPL} è definita nella tabella \ref{tab:facpl_syntax}.
La sintassi è fornita come una grammatica di tipo \ac{EBNF}, dove il simbolo ? corrisponde ad un elemento opzionale, il simbolo $*$ corrisponde ad una sequenza con un numero arbitrario di elementi (anche $0$), ed il simbolo $+$ corrisponde ad una sequenza non vuota con un numero arbitrario di elementi. \par
Al livello più alto c'è il \ac{PAS}, il quale definisce le specifiche del \ac{PEP} e del \ac{PDP}. Il \ac{PEP} è definito semplicemente come un \textit{enforcing algorithm} che sarà applicato per decidere su quali decisioni verrà eseguito il processo di \textit{enforcement}.  \par
Il \ac{PDP} invece è definito come una sequenza (non vuota) di \textit{Policy}, ed un algoritmo di combining che combinerà i risultati di queste \textit{policy} per ottenere un unico risultato finale. \par
Una \textit{policy} può essere una semplice \textit{rule} o una \textit{policy set}, quest'ultima avrà al suo interno altre \textit{policy set} o \textit{rule}, ed in questo modo viene formata una gerarchia di \textit{policy}. \par
Un \textit{policy set} individua un target, che è una espressione che indica il set di richieste di accesso alla quale si applica la \textit{policy}, una lista di \textit{obligations}, che definiscono azioni obbligatorie o opzionali che devono essere eseguite nel processo di \textit{enforcement}, una sequenza di altre \textit{policy}, ed un algoritmo per combinarle. \par
Una \textit{rule} includerà un \textit{effect}, che sarà \permit \ o \deny \ quando la regola è valutata correttamente, un target ed una lista di \textit{obligations}. \par
Le \textit{Expressions} sono formate da \textit{attribute names} e valori (per esempio boolean, double, strings, date). \par
Un \textit{Attribute Name} indica il valore di un attributo che può essere contenuto nella richiesta o nel contesto. FACPL usa per gli \textit{Attribute Name} una forma del tipo \textit{Identifier / Identifier }, dove il primo Identifier indica la categoria, ed il secondo il nome dell'attributo.
Per esempio \textit{Action / ID} rappresenta il valore di un attributo ID di categoria Action. \par
I \textit{Combining Algorithm} implementano diverse strategie che servono per risolvere conflitti tra le varie decisioni, restituendo alla fine un'unica decisione finale. \par
Una \textit{obligation} ha al suo interno un effect, un tipo, ed una azione eseguita dal PEP con la relativa \textit{Expression}. \par
Una \textit{request} consiste di una sequenza di attributi organizzati in categorie. \par
La risposta ad una valutazione di una richiesta \ac{FACPL} è scritta usando la sintassi riportata in Tabella \ref{tab:facpl_context_syntax}.
La valutazione in due step, descritta precedentemente in Sezione~\ref{sec:valutazione_facpl}, produce due tipi di risultati. Il primo è la risposta del \ac{PDP}, il secondo è una decisione, ovvero una risposta del \ac{PEP}.
La decisione del \ac{PDP}, nel caso in cui ritorni \permit \ o \deny, viene associata ad una lista, anche vuota, di \textit{fulfilled obligation}. \par
Una \textit{fulfilled obligation} è una semplice coppia formata da un tipo (M o O) ed una azione i quali argomenti sono ottenuti dalla valutazione del \ac{PDP}. Rappresenta una obligation valutata dal \ac{PEP}

\section{Semantica}
\label{sec:semantica_originale}

Molteplici sono le componenti di \ac{FACPL}, e la semantica ora verrà informalmente analizzata. La semantica formale è presente in \cite{fullfacpl}.
Prima è presentato il processo di decisione del \ac{PDP}, successivamente quello \ac{PEP}. \par
Quando il \ac{PDP} riceve una richiesta, per prima cosa la valuta sulle basi delle \textit{policy} disponibili, successivamente determinerà un risultato combinando le decisioni ritornate da queste \textit{policy} attraverso degli algoritmi di combining. \par
La valutazione della \textit{policy} rispetto alla richiesta comincia verificando l'applicabilità alla richiesta, che è fatta valutando un espressione definita \textit{target}. \par
Supponiamo che l'applicabilità dia esito positivo, nel caso ci sia una \textit{rule} sarà ritornato il valore risultante dalla valutazione di quest'ultima, mentre se c'è un \textit{policy set} il risultato è ottenuto valutando le \textit{policy} contenute all'interno, e combinando i loro valori con uno specifico algoritmo. Successivamente a queste valutazioni viene effettuato il \textit{fulfilment} delle obligation contenute all'interno delle \textit{policy}. \par
Supponiamo ora che l'applicabilità non dia esito positivo, ovvero la valutazione del \textit{target} restituisca \texttt{false}. In questo caso il risultato della \textit{policy} sarà \texttt{not-app}. Mentre se il \textit{target} restituisce un valore non booleano o ritorna un errore il risultato della \textit{policy} sarà \texttt{indet}.

Valutare le espressioni corrisponde ad applicare degli operatori e risolvere i nomi degli attributi che contengono, e di conseguenza ricavarne un valore. \par

La valutazione di una \textit{policy} termina con il \textit{fulfillment} di tutte le obligations  che hanno il valore di applicabilità coincidente con quello ritornato dalla valutazione della \textit{policy}. Quest'operazione consiste nel valutare tutte le espressioni presenti al interno delle \texttt{obligations} coinvolte nel processo. Se ci sarà un errore nel processo di \textit{fulfillment} allora il risultato della \textit{policy} sarà \texttt{indet}, altrimenti il risultato del \textit{fulfillment} sarà uguale a quello della valutazione del PDP. \par
Gli algoritmi di combining hanno lo scopo di combinare le decisioni risultanti dalla valutazione delle richieste in accordo con le policy. Un'altra funzione che hanno è, nel caso nel caso in cui la valutazione finale risulti \permit \ o \texttt{deny}, ritornare le \textit{obligations} coerenti con il risultato della decisione.  
Come ultimo step il risultato del \ac{PDP} viene mandato al \ac{PEP} per l'enforcement.
Il PEP per effettuare questo processo deve eseguire l'azione all'interno di ogni \textit{fulfilled obligation} e decidere come comportarsi per le decisioni di tipo \texttt{not-app} e \texttt{indet.} \par
Per fare questo processo il \ac{PEP} usa delle strategie. In particolare, l'algoritmo \texttt{deny-biased} (rispettivamente, \texttt{permit-based}) effettua l'enforcement dei \permit \ (rispettivamente \deny) solo quando tutte le corrispondenti obligations sono correttamente eseguite, mentre effettua l'enforcement dei \deny \  (rispettivamente \permit) in tutti gli altri casi. Invece, l'algoritmo di base lascia tutte le decisioni non cambiate ma, in caso di decisioni \permit \ e \deny, effettua l'enforcement di \texttt{indet} se ci sarà un errore durante l'esecuzione delle \texttt{obligations}. Questo evidenzia che le \texttt{obligations} non solo influenzano il processo di autorizzazione, ma anche l'enforcement. Gli errori causati dalle \texttt{obligations} con tipo O vengono ignorati.


\section{Tool}
\label{sec:tool}
Il plugin per Eclipse di \ac{FACPL} è stato scritto con l'ausilio di un \textit{Framework} chiamato \textit{Xtext}. Quest'ultimo a sua volta utilizza \textit{Xtend} poiché permette di scrivere regole di traduzione per generare automaticamente codice Java.\par
Attraverso il plugin si riesce a rendere Eclipse un vero e proprio ambiente di sviluppo per \ac{FACPL} in quanto si ottengono funzioni come l'autocompletamento o l'highlighting del codice.\par
L'ambiente di sviluppo con il relativo plugin, le regole di traduzione, e la libreria Java formano insieme la \textit{toolchain} di \ac{FACPL}, mostrata in Figura~\ref{fig:toolChain.png}.

\MyFig{toolChain.png}{ToolChain di FACPL}{1}{h}

Lo sviluppatore che utilizza l'\ac{IDE} di \ac{FACPL} può generare codice Java o \ac{XACML} a partire da politiche scritte in \ac{FACPL}. La traduzione da codice \ac{FACPL} a Java o \ac{XACML} è effettuata attraverso le regole di traduzione scritte in Xtend.

\section{Esempi di politiche} % (fold)
\label{sec:esempio_di_politica_con_facpl}
In questa sezione è analizzata una semplice politica scritta in \ac{FACPL} con delle possibili richieste di accesso, la sintassi delle richieste e delle politiche è leggermente diversa da quella riportata in Tabella \ref{tab:facpl_syntax} in quanto, per questioni di comodità e facilità di lettura del codice, è stata usata quella del plugin.\par
\lstinputlisting[firstline = 1, lastline = 17, language = FACPL, caption = {Esempio di politica in FACPL}\label{lst:facpl_es_cap3}]{./Source/EsempioFacplCap3} 
Con il Codice \ref{lst:facpl_es_cap3} si vuole ottenere lo scopo di regolare l'accesso ad una risorsa chiamata 458. In questo caso gli utenti che hanno ruolo \textit{GUEST} non possono accedere, mentre gli \textit{ADMINISTRATOR} si, fatta eccezione per l'utente Peronio, che qualunque ruolo abbia può accedere.
Le richieste effettuate al sistema vengono mostrate in Codice~\ref{lst:facpl_es_cap3_req}, e sono tre. La prima proviene dall'utente Gianfabrizio che fa parte degli \textit{ADMINISTRATOR}, la seconda e la terza rispettivamente dall'utente Gianpietro e Peronio che fanno entrambi parte dei \textit{GUEST}. 
\lstinputlisting[firstline = 21, lastline = 38, language = FACPL, caption = {Richieste per Codice~\ref{lst:facpl_es_cap3} }\label{lst:facpl_es_cap3_req}]{./Source/EsempioFacplCap3} 

\begin{table}[H]
\centering
\caption{Risultati delle richieste}
\label{tab:risultati_richieste}
\hrule
\begin{tabular}{lll}
 & \textbf{Risultato} & \textbf{Obligation} \\
\textbf{Richiesta 1} & \textit{PERMIT} & PERMIT M action1({[}GIANFABRIZIO{]}) \\
\textbf{Richiesta 2} & \textit{DENY} & DENY M action2({[}GianPietro{]}) \\
\textbf{Richiesta 3} & \textit{PERMIT} & PERMIT M action1({[}PERONIO{]})
\end{tabular}
\hrule
\end{table}
In Tabella \ref{tab:risultati_richieste} sono riassunti i risultati delle richieste. Ovviamente alla prima richiesta il risultato è \permit, in quanto l'utente è un amministratore. Alla seconda richiesta il risultato è \deny \ poiché l'utente è un ospite, mentre alla terza, nonostante l'utente faccia parte dello stesso gruppo del secondo riesce ad ottenere risultato \permit \ per via della regola che considera il suo nome. \par
\ac{FACPL}, come mostrato dall'esempio, permette di fare richieste ed ottenere delle risposte, ma queste richieste sono totalmente indipendenti l'una dall'altra, quindi l'ordine di esecuzione non avrebbe influenzato in alcun modo il risultato finale.
In sezione \ref{sec:usage_control} sono stati introdotti due esempi i quali non possono, per ora, essere implementati in FACPL poiché manca quest'aspetto che crea dipendenza tra le richieste.
Per creare questa dipendenza tra richieste è necessario che il sistema di controllo agli accessi riesca a tenere traccia degli avvenimenti precedenti, perciò si inizia a parlare di un nuovo concetto che può essere assimilabile ad uno stato. Nel Capitolo~\ref{cap:usagecontrolfacpl} e ~\ref{cap:estensione_libreria} l'obiettivo è proprio permettere a \ac{FACPL} questo tipo di valutazione.

% section esempio_di_politica_con_facpl (end)