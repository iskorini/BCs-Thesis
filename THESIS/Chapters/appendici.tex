\appendix 
\myChapter{Codice completo}
\label{cap:appendiceA}
\section{Capitolo 4}
\lstinputlisting[language = FACPL, caption = {Secondo Esempio Completo}\label{lst:SecEsCompl}]{./Source/second_example_facpl}
\section{Capitolo 5}
\label{sec:Cap5app}
\subsection{Status e Status Attribute} % (fold)
\label{sub:status_e_status_attribute}

% subsection status_e_status_attribute (end)
\myjava{status.java}{Stralcio della classe Status}
\myjava{SA.java}{Costruttori di Status Attribute}

\subsection{Implementazione dei comparatori sugli Status Attribute} % (fold)
\label{sub:implementazione_dei_comparatori_sugli_status_attribute}
\myjava{equal.java}{Classe che implementa Equal}


\subsection{Funzioni per la modifica degli Status Attribute} % (fold)
\label{sub:funzioni_aritmetiche_per_la_modifica_degli_statusattribute}
\myjava{valutatore.java}{Metodo implementato dall'interfaccia}


\subsection{Obligations e PEP} % (fold)
\label{sub:obligations_e_pep}

\myjava{absObl.java}{Parte rifattorizzata del metodo che si occupa del fulfilling}
\myjava{oblStat.java}{CreateObligation nelle status}
\myjava{oblNorm.java}{CreateObligation nelle normali}
\myjava{PEP.java}{Discharge delle Fulfilled Obligation di stato}

\subsection{Esempio}
\label{sub:esempio_5}

\myjava{richiestajavaprimoesempio.java}{Policy StopRead}
\myIjava{richiestajavaprimoesempio.java}{Policy StopRead}{92}{122}{polstopread}
